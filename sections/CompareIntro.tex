Although all the colliders we reviewed have their advantages and disadvantages, we have decided to look into two in more depth: ILC and CLIC. 

These colliders have a lot in common. They are both lepton colliders, meaning that they are able to produce clean collisions as the particles do not divide after the collision, and therefore lead to precise measurements. This is in contrast to hadron colliders such as the LHC, which produce collisions in which many particles are released, and are called ‘dirty’ collisions. We have decided to choose lepton colliders, leading to clean collisions, as these are best to determine specific properties of particles such as top, bottom and Z quarks, as well as the Higgs boson. This also allows us to continue the momentum following the discovery of the Higgs boson, by probing it further now that we know its energy level. A lepton collider means that a specific energy range can be chosen, in contrast to colliders such as the LHC where the actual energy range of any collision is can be unknown.

They are also both linear colliders. Although circular colliders such as TLEP can generate large energies, they lose energy to synchrotron radiation, which does not occur with a linear accelerator. This problem could be solved by using a muon collider, as muons have such a large mass that any synchrotron radiation is negligible. However, any plans for muon colliders are far from complete, and the technology needed is unfeasible in the near future. They also both have high maximum energies of 1TeV and 3TeV respectively. This may not seem high in comparison to the 100TeV of the VLHC which consists of TLEP using LHC as an injector; however that project would only begin after the LHC’s lifetime which is a significant amount of time for the scientific community to wait – approximately 40 years.

In addition, ILC and CLIC are the most feasible, as their plans are the most finalised and viable. This is especially true in comparison to colliders such as muon colliders which are completely unfeasible so far and will be for the near future. They use technology which has either already been established or is in the final stages of testing, and they have already found sites to use. They also have plans for funding, whether they are being funded partly by governments, i.e. Japan for ILC, or will be funded by CERN, i.e. CLIC.