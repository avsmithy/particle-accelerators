Compact Linear Collider (CLIC) is an anticipated high energy and high luminosity $e^+ e^-$ linear accelerator. Positrons and electrons are fundamental particles that annihilate to produce a burst of energy known with great accuracy, making a cleaner collision. This makes it easier to pick out relevant data from background information.

The intense energy coalesces into numerous new particles thus opening a doorway to novel physics. CLIC would be used to compliment the Large Hadron Collider (LHC) by probing areas of interest highlighted by LHC research with a higher degree of precision than is provided by proton- proton collisions. According to CERN, CLIC will not only offer the opportunity of precise measurements of mass and couplings of new particles discovered at the LHC, but will also extend the discovery reach for particles that suffer from low production cross-section at the LHC. \cite{LHC:CP:Higgs}

\subsubsection{Design}

As a linear collider, CLIC will use two linear accelerators, each with a drive and main beam; one accelerator is for electrons, while the other is for positrons. The two accelerators will point directly at each other, synchronously shooting beams of particles that collide head on at the centre of a detector.

The detector is filled with consecutive detection layers. The particles cross the different detector layers in which signals generated by their passage produces a collection of thorough knowledge about each particle, such as its energy, electric charge and type.

CLIC is designed to have two detectors, but only one is to be used at a time. This means that the detectors will lie on a large `push\textendash pull' platform that is moved across the collision point. With two detectors, one will be able to verify the results of the other, which will be imperative for confirming new findings.

\subsubsection{Physics Potential}
 
The Higgs physics potential at CLIC would be extraordinary, due to the large energy range of CLIC. Since being discovered at the LHC, the Higgs boson requires a more precise study at different stages of energy production and decay modes. At CLIC, this would require approximately 125 GeV of energy. At an energy range of $\leq$ 500 GeV at CLIC, the Z recoil from HZ production could be measured, offering mass determination (only measured using lepton colliders) and model-independent coupling. At a higher energy ($\sim$1 TeV) at CLIC, the top Yukawa coupling and tri-linear Higgs self-coupling could be measured using double Higgs production.
At the highest energy at CLIC of $\sim$3 TeV, the branching fraction of rare processes (e.g. decay into muons) would be determined.
 
There is a top quark physics potential at CLIC, where there may be the possibility of a hint to Beyond Standard Model physics, this is because the heaviest Standard Model particle couples most strongly to Higgs field. The top quark mass can be precisely be determined and a direct reconstruction of top quarks from decay products at energies above the production threshold will be produced.
 
Theoretically, decay chains of strongly interacting supersymmetric particles can abundantly produce heavy sleptons, charginos and neutralinos however sometimes these chains are not able to access all states. Instead, CLIC can thoroughly look for new particles with electroweak charges through exploring the TeV region. \cite{CLIC:Concept}

``The precise mass and coupling measurements that can be performed at CLIC allow us to address fundamental questions integral to the mechanism of supersymmetry, aspects of unification, and the feasibility of the lightest supersymmetric particle as a dark matter thermal relic.'' \cite{CLIC:Concept}
 
\subsubsection{Feasibility}

The CLIC Test Facility (CTF3), CERN has already started testing the feasibility of requisite technologies: power extraction, two-beam acceleration and recombination, and beam stabilisation. \cite{CLIC:DriveBeam}

CLIC requires an accelerating gradient of 100 MV/m; however, superconducting accelerating cavities have an accelerating gradient restriction of approximately 60MV/ (where above this limit the superconducting properties are lost). Therefore CLIC will require a room temperature, less power-efficient accelerating cavity. But for the same collision energy it allows a shorter accelerator length than ILC. For CERN the total average power consumption is 200 MW during peak intervals whilst LHC was in operation, CLIC power consumption is estimated to be 415 MW, at 3 TeV, which is more than double that.\cite{CERN:Powering}

During tests, the inside of the room temperature copper cavities become damaged: this needs to be resolved (although plating with tungsten or molybdenum may be a solution). Another obstacle to CLIC would be to narrow the beams to a nanometer at the collision point \cite{Nuclear:CTF3}. Stability and accuracy is of utmost importance, as minute errors in parameters such as phase at source and beam charge will have a collective effect on the luminosity of the beam. \cite{CLIC:DriveBeam}

At the same time, new particle detectors will need to be designed to be able to cope with the physics at 3 TeV of centre-of-mass-energies and be able to operate efficiently in the CLIC machine environment. \cite{CLIC:Concept}

A major challenge for CLIC will be the manufacture of huge numbers of high strength focusing quadrupole magnets for the drive beam. Manufacturing rates of about 50 magnets per day will be required, and this is way beyond present capabilities. \cite{CLIC:STFC}