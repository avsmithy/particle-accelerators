The International Linear Collider (ILC) and Compact Linear Collider (CLIC) missions are to explore the fundamental composition of nature, by observing reactions of fundamental particles at the extremely high energies attainable in these superconducting linear colliders. In order for the ILC and CLIC to achieve these goals, they need to develop cutting-edge breakthroughs in numerous systems and technologies crucial for the designing, engineering and constructing of both ILC and CLIC. With the required technology created, there is a potential to bring benefits to many different areas of society, which is to be further discussed in this section. \cite{ILC:SpinOffReport}

\subsubsection{Medicine}
 
\paragraph{Positron Emission Tomography (PET).}
 
PET is an outcome of antimatter physics research, which has become an integral part of medical diagnostics, which allows views of chemical processes within live organs, once thought of as unattainable. As well as PET, proton therapy is a powerful treatment method, which needs costly and heavy equipment to deliver a targeted and concentrated dosage of protons with precision to a cancerous tumour site. However with the new ILC superconducting accelerating technologies, it is possible, not only to reduce power consumption, making it cheaper, but to downsize the equipment to make it lighter. Radiation therapy could become more accurate and therefore less damaging to healthy tissue by synchronising with the patient's breathing cycle.
 
\paragraph{Photo-cathode Electron Guns.}

At the collision point of ILC and CLIC, high luminosity is required, for which a large current needs to be generated. Additionally, it is necessary to generate low emittance and ultra short electron beams during beam generation. Photo-cathode guns offer small thermal emittance, though electrons have almost no speed at the time of generation, which can, however, be compensated by applying an accelerating electric field immediately after beam generation in order to increase the speed of the electrons. Therefore, photo-cathode electron guns are suitable as a high-quality electron beam source. Photo-cathode guns have acted as an integral source of electron beams in various processes, such as in Intensity Modulated Radiation Therapy (IMRT), a modern radiotherapy method.

In the case of normal radiotherapy, at the time of irradiation, as the normal tissues are irradiated together with the cancer cells, the radiation dosage is limited by the amount of radiation normal tissues can tolerate. The consequence is that the irradiation to the cancer cells tends to be insufficient. However, in the case of IMRT, radiation is delivered at numerous intensities coming from various directions based on optimal radiation conditions, which are derived from a computer simulation. This provides a sufficient amount of radiation, which is provided only for the cancer cells, which causes no damage to normal tissue. In order to generate radiation with accurate and optimum intensity electron beams, the ILC and CLIC photo-cathode electron gun technology is needed.
 
The ILC and CLIC superconducting technology could be adapted to produce monochromatic X-rays for medical diagnoses and treatment, enabling radically new probes of biological processes and tissue protein structure, and help develop new medicines.
 
\subsubsection{Computing}

The data transfer rates from experiments like those at the ILC, CLIC and the Large Hadron Collider are enormous – comparable to those for all the world's telecommunications put together. The latest computer and communications technologies and the advanced Grid data flow management software developed by particle physicists are essential to assist in analysis of images. The MammoGrid database developed in European laboratories now uses this technology for medical image analysis, moving beyond diagnosis to monitor therapy and disease progression \cite{CERN:MammoGrid}. This remote sharing of diagnoses allows for second opinions between clinicians and allows for potential knowledge in early diagnosis of breast cancer development and treatment. A repository with 30,000 mammograms is now accessible, and in turn helps in saving lives. \cite{ILC:WhyNeed}
 
Highly advanced simulation codes for ILC and CLIC technology are currently being developed. The electromagnetic field analysis codes are needed for the accelerator beam track analysis. This software will be hugely applicable to various industries such as the evaluation and tuning up of electron beam based industrial equipment such as scanning electron microscopes (SEM). As well as that, in the medical therapy field, the software is already being used to evaluate the dosage of radiation in boron neutron capture therapy (BNCT) and to proton and heavy ion therapy for cancer.

\subsubsection{Environment}

Superconducting technology could also be used to minimise cost of disposal of nuclear waste through producing intense gamma rays to characterise the composition of nuclear waste. As it is approximated that disposal of nuclear waste in Japan alone will reach a cost of \$30 billion by 2020 \cite{Numo:DisposalCost}, the search for a low cost way of disposing nuclear waste is being sought through superconducting technology.

By identifying nuclides and respective concentration, proper segregation can be achieved through non-destructive means, where cost of disposal is minimised. In order to solve this problem, the application of photonuclear resonance scattering of laser Compton gamma rays to radioactive nuclei is used to segregate nuclear waste. The laser Compton gamma ray can identify almost all nuclides including stable isotopes, enabling an appropriate segregation of wastes.

The method of giant nuclear resonance caused by bombarding high intensity gamma ray beams on contained nuclei in order to transmute to stable nuclei has been proposed. Experiments are under way for this purpose using large-scale accelerator facilities.

As well as positive contributions to nuclear waste, monitoring technologies for precise beam control could be used as a precise early warning system for seismic activity. Technology for this is still being proposed. \cite{ILC:WhyNeed}

\subsubsection{Tools for the future}

Particle beams at both ILC and CLIC need regular monitoring with fast and accurate corrections. Tools in both schemes developed for this reason will help design highly integrated circuit fabrication procedures, helping boost industrial products and processes at a very small scale (nanometer). Due to improved technology for electron lithography, computers and laptops could become even more lightweight and compact. The electron beam lithography from beam control systems could also produce efficient PC chips.

Techniques used to give accelerator cavities the attractive polish could influence technologies in the metal industry which are better understood and cheaper. Moreover, the magnetic disk industry will be revolutionised by new electron microscopes from electron sources developed for ILC and CLIC. Customs officers will also benefit by particle physics, as detectors developed for particle collisions will be able to analyse contents of cargo containers. \cite{ILC:WhyNeed}

\subsubsection{ILC Technology and other sciences}

Energy Recovery Linacs (ERLs) have an impact on many areas of studies, significantly expanding capabilities for studies in materials science, structural biology, nuclear science, environmental studies and chemistry. The superconducting technology from ILC and CLIC will advance work on the ERLs, allowing for a substantial amount of cost savings. Projects like Free-Electron Lasers (FELs) are being initiated and built in countries such as Japan, US and Germany based on linear collider research. As well as this, light sources have been able to influence important advances within numerous sciences over the past few years, leading to many applications. Researchers in the US at the Advanced Light Source have been able to solve the structure of the avian flu virus and analyse its interaction with human receptors.

ILC and CLIC technology can also be used to accelerate protons and nuclei, where proton accelerators used for strong spallation neutron sources can be applied to a wide range of studies on biological properties. Numerous areas of study such as medical implants, lighter airplanes, and corrosion control within material science will also be positively impacted. \cite{ILC:WhyNeed}

\subsubsection{People and skills}

ILC and CLIC play an important role in attracting the new generation of intrigued engineers and scientists crucial to the development of society. Over the past forty years, particle physics theory and experiment has become a subject of great fascination internationally. Scientists from all over the world come together to collaborate, developing close, cooperative business relationships, which have influenced international relations, especially when scientists hold high positions in their own homeland. 

Bringing a new range of ideas, innovative and highly qualified engineers and scientists have had a more immediate effect on industrial, commercial and medical sectors of society. They have been given the opportunity to bring new ideas to the table, to apply all new particle physics technology to a wide range of problems in every day life. 

Particle physics has and will continue to encourage young people to seek out careers in technology and science. The continuation of particle physics and specifically collider physics will continue to physically widen our technology in society as described by the above spin offs, and will attract the workforce of the future, equipping them with the passion and perseverance to develop new acceleration and detector prototypes for greater probing into the mysteries of the universe. \cite{ILC:WhyNeed}

\subsubsection{Critical Assessment comparing CLIC and ILC Spin Offs}

CLIC has a higher energy range than ILC, however as they are both linear accelerators, the spin offs do not differ, as they do not rely on an energy range higher than 1 TeV. However, any other spin offs relying on the acceleration gradient technology of CLIC have not been explored yet, as the project is a lot less `ready to go' than ILC.