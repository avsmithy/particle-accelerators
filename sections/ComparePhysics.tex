\subsubsection{Introduction}
ILC and CLIC are both designed to achieve a very different energy reach. The energy reach is another way of referring to the centre\textendash of\textendash mass energy of the colliding particles. The energy reach of the ILC is 0.5 TeV with a possible upgrade to 1 TeV; this is small in comparison to CLIC's energy reach of 3 TeV. However the sheer size of the energy reach is not the only factor that will determine which is more valuable as a future particle collider. Since CLIC uses the same particles for collision as the ILC, but with a potentially higher energy range, all the interactions that can occur in the ILC can do so in CLIC.

%Other factors include; design considerations, e.g. whether the cost of running the higher energy CLIC is worth the output; whether it is realistic to assume the beams can achieve the energies that they claim; and whether or not there is a need for the higher energies that CLIC will provide.  

\subsubsection{Higgs}

In the $\leq$ 1 TeV range, both the ILC and CLIC will be able to precisely measure the Higgs boson at $\sim$126 GeV. As both ILC and CLIC are $e^+e^-$ colliders the interactions will be much cleaner than those at CERN's LHC and thus allow much greater precision measurements of the Higg's mass, spin and interaction energy. This will allow an in depth analysis of the Higgs, which is currently the driving force behind the construction of a particle collider. 

Furthermore they will be able to differentiate between types of Higgs bosons (if they exist) and determine the coupling of the Higgs to itself (as described by the Standard Model), thereby proving (or disproving) that this is the Standard Model Higgs. Yukawa coupling, branching fractions of rare processes and extended Higgs sectors will also be observable at the 1 TeV limit of ILC. There is no major advantage of CLIC over ILC in terms of direct investigation into the Higgs.

\subsubsection{Top Quarks}

Top quarks are the heaviest subatomic particles known to exist and have a predicted lifetime in the order of $10^{-25}$ s. Due to its large mass, which means a greater predicted coupling to the Higgs field, it is frequently investigated to see how the Higgs boson interacts with matter. Precise measurement of the top quark mass will allow competing theories of the Higgs to be dismissed.

The Tevatron and LHC have measured the mass to $\pm 1$ GeV uncertainty, but improved results are not expected to come from the LHC. The mass and decay width of the top quark must be known to a greater precision than it is now and both CLIC and ILC can achieve this. There is a minor advantage in statistical precision for CLIC due to its higher luminosity; however this is still well below the systematic uncertainties for both machines. \cite{CERN:TopQuark, CLIC:TopQuark}

\subsubsection{Supersymmetry}

Although support of the SUSY model being correct has declined recently, as no evidence for it has been observed, both ILC and CLIC will attempt to investigate it. If SUSY does exist, the LHC will likely find evidence for it in the coming years but will only be able to meaningfully explore the superpartners of first generation quarks and gluons. ILC and CLIC are much better suited to investigate the superpartners of electroweak particles due to their cleaner interactions.

The ILC will be able to explore superpartners of the Higgs boson and first and second generation quarks and leptons. There is a {\em possibility} that third generation particles will lie within ILC's reach, however full discovery of SUSY would require a multi-TeV machine. CLIC has been specifically designed for the sector and would be considerably better than ILC, being able to explore all fermion superpartners as predicted in the current model. \cite{CLIC:Concept}

\subsubsection{Dark Matter}

Dark matter is an observed phenomenon in cosmology with many different theories attempting to explain it, with the most widely regarded being Weakly Interacting Massive Particles (WIMPs). Should WIMPs be discovered, they can be compared to cosmologically observed dark matter to link a new sector of physics from the small to large scale. The LHC could produce WIMPs, but detection would be obscured by the large background.

The ILC would be able to discover or dismiss a class of dark matter when operating at its maximum (1 TeV) level. CLIC would be able to measure the masses of the same particles to a greater precision and look at particles with twice the mass than those at ILC. This investigation of WIMPs is all dependent on discovery of supersymmetry, as they are only predicted in SUSY.

\subsubsection{Luminosity}

When contemplating the energy reach it is important to consider the luminosities of each collider, as this will define how many events are detected. Luminosity is a challenge for all linear colliders and one that is overcome by bunching the particles to reduce the interaction cross section. CLIC is envisioned to have a luminosity of approximately $10^{35}$ $cm^{-2} s^{−1}$ \cite{CLIC:Luminosity}, this is ten times larger than ILCs intended luminosity of $10^{34}$ $cm^{-2} s^{−1}$ \cite{IOP:ILC}. It is therefore clear that as well as a higher energy reach, CLIC also expects a higher luminosity meaning more interactions can be observed.


% \subsection{Extra Dimensions? Z'? Contact Interactions?}

%\subsubsection{Relative Design}
%When comparing the colliders' energy reach the first aspect to be considered is the design i.e. how the colliding energies will be achieved. It is important to understand the fundamental differences and similarities in the design in order to assess which is a more robust choice of collider.
%
%The first item to be compared is the overall design of the two colliders. Both colliders are linear with one key difference; CLIC uses a two-beam design and ILC uses the standard one beam design. The reason for this difference is the gradient required to accelerate the beams. The ILC requires a gradient of 31.5 MV/m \cite{ILC:TechnicalDesignReport} in order to achieve its target energy reach of 0.5 TeV. This can be achieved with superconducting cavities that are currently in production and thus only one beam is needed. In comparison CLIC requires a much larger gradient of 100 MV/m to achieve its energy reach of up to 3 TeV \cite{CLIC:ParameterList}. After approximately 60 MV/m the superconducting cavities lose their superconductive properties and thus an upper limit is imposed on the acceleration gradient they can provide. This then requires a two-beam design in which `drive beams' that run parallel to the main linacs build up power and then transfer the particles to the main linacs. When comparing the overall design it is clear that ILCs lower energy beam is much easier to achieve than CLIC's two-beam system.
%
%The last facet to study when comparing the energy reach of the two colliders is the power consumption of each collider when active. ILC will be consuming 200-250 MW when running and 50 MW during downtime \cite{ILC:Director}. The total power consumption of CLIC is estimated to be 582MW \cite{CLIC:PowerConsumption}, most of this power is consumed in the drive beams $\sim$305MW will be used here. Thus it is clear that the power that ILC will consume is significantly less than that of CLIC, this is to be expected, as the energy is so much lower. This will of course mean that the costs of running CLIC will be higher even if the capital cost of the construction is equivalent to ILC and is thus a significant factor.
%
%Overall it is clear that the design of ILC is far simpler than that of CLIC and will consume less power whilst in operation. However the greater energies of CLIC and the associated luminosity may be of greater use to the physics community. To discuss this further a brief feasibility assessment of the designs must be made on top of comparing the necessity of higher energy.

%\subsubsection{Realism of Energies}
%Currently nearly all aspects of ILC are feasible, as it does not use any new technologies. The only parts that there is issue with is the marx modulators that will power the klystrons, as these have not been made on the industrial scale that is required for the construction of this collider \cite{ILC:TechnicalDesignReport}. This is only a small issue and will be easily overcome if ILC is commissioned.
%
%In comparison CLIC is far from ready to be built. The two main issues facing the CLIC are the accelerator gradients that are required to produce the 3TeV energy and the two-beam accelerator structure that is being proposed. These challenges will require at least ten years worth of research and development to overcome and thus puts the projects construction date much further in the future than that of ILC. Though at this stage it is firmly believed that following the advancements in technology that will occur over the next ten years that the design of CLIC will become feasible.
%
%Assuming a best-case scenario where CLIC is feasible in ten years time then the question posed is: do we wait for a higher energy collider or do we build ILC immediately? This question is hard to answer outright and leads to a further question: what is the benefit of waiting for a higher energy collider?

\subsubsection{The Need for Higher Energies}
\label{higherEnergies}
Due to the higher energies available at CLIC the centre of mass energy of the interacting particles could actually be greater than that currently being produced at the LHC. This is due to the fact that the LHC is colliding protons that are composed of three quarks; this leads to the 8 TeV centre of mass energy \cite{ATLAS:8TeV} of the protons being divided amongst these quarks, leading to a colliding energy of around 2.7 TeV, which is less than the proposed 3 TeV of CLIC. Therefore as well as being able to make precision measurements CLIC will also be pushing the energy boundary and possibly discovering new particles that could confirm the Standard Model further.

When comparing the need for the energies we see that CLIC can produce the same results as ILC whilst also pushing the energy frontier. It is thus fair to conclude that the ILC would be a precision measurement tool used to explore the Higgs boson in great depth and in a nearer future to CLIC, whilst CLIC is both a precision measurement tool and an energy frontier pushing collider that could yield the discovery of new and more exotic particles. An argument could be made that neither collider is of high enough energy, and that waiting for a new collider design while continuing the experiments at the LHC is the best course of action. However this overlooks the need to study the Higgs particle in greater detail.

	
\subsubsection{Conclusion}
The luminosity and energy reach at CLIC is superior to ILC making observation of new interactions more achievable; it can be used for a larger range of experiments and will impact the physics community in a more profound manner. For currently known particles (e.g. Higgs boson and top quarks), it appears that CLIC has an advantage over the ILC. In terms of SUSY and Dark Matter, CLIC does have a reasonable advantage, but these theories may be completely wrong. Until there is direct evidence for SUSY and/or Dark Matter (likely from the LHC), it is hard to use SUSY/DM to justify CLIC over ILC.