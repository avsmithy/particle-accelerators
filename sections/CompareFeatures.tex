\subsection{Energy Reach Evaluation and Comparison}

\subsubsection{Introduction}
ILC and CLIC are both designed to achieve a very different energy reach. The energy reach is another way of referring to the centre of mass energy of the colliding particles; in both ILC and CLIC these colliding particles are an electron and a positron.  The energy reach of the ILC is 0.5 TeV with a possible upgrade to 1 TeV this is small in comparison to CLIC's energy reach of 3 TeV. However the sheer size of the energy reach is not the only factor that will determine which is more valuable as a future particle collider.

Other factors include; design considerations, e.g. whether the cost of running the higher energy CLIC is worth the output; whether it is realistic to assume the beams can achieve the energies that they claim; and whether or not there is a need for the higher energies that CLIC will provide.

\subsubsection{Relative Design}
When comparing the colliders' energy reach the first aspect to be considered is the design i.e. how the colliding energies will be achieved. It is important to understand the fundamental differences and similarities in the design in order to assess which is a more robust choice of collider.

The first item to be compared is the overall design of the two colliders. Both colliders are linear with one key difference; CLIC uses a two-beam design and ILC uses the standard one beam design. The reason for this difference is the gradient required to accelerate the beams. The ILC requires a gradient of 31.5 MV/m \cite{ILC:ReferenceDesignReport} in order to achieve its target energy reach of 0.5 TeV. This can be achieved with superconducting cavities that are currently in production and thus only one beam is needed. In comparison CLIC requires a much larger gradient of 100 MV/m to achieve its energy reach of up to 3 TeV \cite{CLIC:ParameterList}. After approximately 60 MV/m the superconducting cavities lose their superconductive properties and thus an upper limit is imposed on the acceleration gradient they can provide. This then requires a two-beam design in which `drive beams' that run parallel to the main linacs build up power and then transfer the particles to the main linacs. When comparing the overall design it is clear that ILCs lower energy beam is much easier to achieve than CLICs two-beam system.

When contemplating the energy reach it is important to also consider the luminosities of each collider, as this will define how many events are detected. Luminosity is a challenge for all linear colliders and one that is overcome by bunching the particles to reduce the interaction cross section. CLIC is envisioned to have a luminosity of approximately 1035 cm$^{-2}$s$^{−1}$ \cite{CLIC:Luminosity} this is ten times larger than ILCs intended luminosity of 1034 cm$^{-2}$s$^{−1}$ \cite{IOP:ILC}. It is therefore clear that as well as a higher energy reach CLIC also expects a higher luminosity meaning more interactions can be observed.

The last facet to study when comparing the energy reach of the two colliders is the power consumption of each collider when active. ILC will be consuming 200-250 MW when running and 50 MW during downtime \cite{ILC:Director}. The total power consumption of CLIC is estimated to be 582 MW \cite{CLIC:PowerConsumption}, most of this power is consumed in the drive beams $\sim$305 MW will be used here. Thus it is clear that the power that ILC will consume is significantly less than that of CLIC, this is to be expected, as the energy is so much lower. This will of course mean that the costs of running CLIC will be higher even if the capital cost of the construction is equivalent to ILC and is thus a significant factor.

Overall it is clear that the design of ILC is far simpler than that of CLIC and will consume less power whilst in operation. However the greater energies of CLIC and the associated luminosity may be of greater use to the physics community. To discuss this further a brief feasibility assessment of the designs must be made on top of comparing the necessity of higher energy.

\subsubsection{Realism of Energies}
The next issue to discuss is how realistic the energy reaches the colliders claim are. This will include a brief overview of the feasibility of the design as a whole, as an in depth feasibility comparison will be made later in the report. % todo check this

Currently nearly all aspects of ILC are feasible, as it does not use any new technologies. The only parts that there is issue with is the marx modulators that will power the klystrons, as these have not been made on the industrial scale that is required for the construction of this collider \cite{ILC:ReferenceDesignReport}. This is only a small issue and will be easily overcome if ILC is commissioned.

In comparison CLIC is far from ready to be built. The two main issues facing the CLIC are the accelerator gradients that are required to produce the 3TeV energy and the two-beam accelerator structure that is being proposed. These challenges will require at least ten years worth of research and development to overcome and thus puts the projects construction date much further in the future than that of ILC. Though at this stage it is firmly believed that following the advancements in technology that will occur over the next ten years that the design of CLIC will become feasible.

Assuming a best-case scenario where CLIC is feasible in ten years time then the question posed is: do we wait for a higher energy collider or do we build ILC immediately? This question is hard to answer outright and leads to a further question: what is the benefit of waiting for a higher energy collider? This is what is explored in the final part of this section. % todo use reference

\subsubsection{The Need for Higher Energies}
\label{higherEnergies}
The final area that should be explored when considering energy reach is the need for that energy reach. It may be that there is no significant advantage to having CLIC's 3 TeV compared to ILC's 0.5-1 TeV, in which case ILC would be the clear choice. This is clearly very similar to the goals of the two colliders and thus the physical significance of the energies will be discussed rather than what the individual colliders hope to achieve; as this will be explored in a later section. % todo use refence

The first observation to be made is that both of the collider energies are capable of producing Higgs particles the energy of which is $\sim$126 GeV \cite{CERN:Higgs}. As both colliders are electron positron colliders the interactions will be much cleaner than those at CERN's LHC and thus allow much greater precision measurements of the Higgs particle's interaction strength, spin and mass. This will allow an in depth analyses of the Higg's particle and is currently the driving force behind the construction of a particle collider.

Another area being explored is supersymmetry, with the search for the lightest supersymmetric particles being a continuing endeavour. These supersymmetric particles were expected to be found at CERN's LHC, however currently they have not been observed. It is possible that they will be seen at the ILC due to the cleaner interactions that are possible though as the energies at the ILC are not as higher as those at the LHC it is not likely. Conversely the higher energies available at CLIC make it a more viable candidate for the production and observation of these supersymmetric particles. This is because the centre of mass energy of the interacting particles could actually be greater than that currently being produced at the LHC. This is due to the fact that the LHC is colliding protons that are composed of three quarks; this leads to the 8 TeV centre of mass energy \cite{ATLAS:8TeV} of the protons being divided amongst these quarks. Thus leading to a colliding energy of around 2.7 TeV, which is less than the proposed 3 TeV of CLIC. Therefore as well as being able to make precision measurements CLIC will also be pushing the energy boundary and possibly discovering new particles that could confirm the standard model further.

Overall when comparing the need for the energies we see that CLIC can produce the same results as ILC whilst also pushing the energy frontier. It is thus fair to conclude that the ILC would be a precision measurement tool used to explore the Higgs boson in great depth and in a nearer future to CLIC, whilst CLIC is both a precision measurement tool and an energy frontier pushing collider that could yield the discovery of new and more exotic particles. An argument could be made that neither collider is of high enough energy, and that waiting for a collider whilst continuing the experiments at the LHC is the best course of action. However this overlooks the need to study the Higgs particle in greater details, as at the LHC this is hard to do due to the noise created when the protons interact; this problem is overcome with the cleaner collision occurring within ILC and CLIC.
	
\subsubsection{Conclusion}
After having explored the design of the two colliders it is clear that ILC's proposed design is far simpler than that of CLIC thus it will be much more likely for ILC to produce the 0.5-1 TeV that it claims. However the luminosity is superior at CLIC making observation of interaction considerably more achievable. As well as having a much more complicated design CLIC also has feasibility issues that will take years of research and Development to overcome, in stark contrast to this ILC has next to no obvious feasibility issues and is ready to be constructed. Finally after comparing what could be achieved at the proposed energies CLIC was found to be a much more versatile collider; not only could it be used to precisely measure the same interactions as ILC it could also be used to push the energy frontier and discover new particles.

In conclusion the energy reach of CLIC is superior to that of the ILC; it can be used for a larger range of experiments and will impact the physics community in a more profound manner. However accompanying the greater energy reach is years of research and development in which time the LHC may find particle that require an even larger energy reach to be examined.
