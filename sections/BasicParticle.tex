\subsection{Standard Model}
The Standard Model of particle physics is a theoretical framework which describes the interactions of all known fundamental particles via the electromagnetic, weak, and strong interactions. According to our understanding, elementary particles can be divided into quarks and leptons, and further classified according to tiers, or ``generations'', with all stable particles belonging to the first generation. The three fundamental forces described by the Standard Model \textendash electromagnetism, weak, and strong forces \textendash are all carried by bosons: massless photons (electromagnetic), massive W and Z bosons (weak), and gluons (strong) \cite{CERN:SM:Online}.

The Standard Model is not a complete theory of fundamental interactions: it is considered to be incompatible with Einstein's theory of general relativity — because general relativity is space-invariant whereas, within the Standard Model, Minkowski space is the fixed background space-time \cite{Colosi:2005:CQG}. We also know from cosmology that the Standard Model accounts for just 4\% of the energy present in the universe — the other 96\% is accounted for by dark matter and dark energy, neither of which is explained by current physics\cite{Krauss:2009:Conference}. Additionally, neutrino oscillation experiments have shown that neutrinos have mass \cite{Fukuda:1998:Kamiokande}, but they are not predicted to have mass by the Standard Model in the same way as other fundamental particles and likely do not acquire mass via the Higgs mechanism. The Standard Model also fails to account for the observed asymmetry of matter and antimatter in our universe \cite{Sather:1999:MatterA}.

\subsection{Symmetry Breaking}
Symmetry breaking refers to a phenomenon where small changes in a system determine which branch of a bifurcation is taken, thereby deciding the fate of the system. There are two different types of symmetry breaking: explicit symmetry breaking, where ``the dynamical equations are not manifestly invariant under the symmetry group considered'' \cite{Brading:2013:Symmetry}, i.e. the Lagrangian contains terms which break the symmetry under consideration; and spontaneous symmetry breaking, where ``solutions exist which are not invariant under the action of this symmetry without any explicit asymmetric input''. This type of symmetry breaking emerged in the study of condensed matter physics, wherein the observation that below the Curie temperature the spins of magnetic dipoles all align in one direction breaks the predicted rotational symmetry for the system \cite{Brading:2013:Symmetry}. The Higgs mechanism by which particles gain mass is one example of spontaneous symmetry breaking in particle physics: below some critical temperature, the electroweak symmetry is broken, and the W and Z bosons acquire masses. Without this spontaneous symmetry breaking via the Higgs mechanism, the W and Z bosons \textendash which we know experimentally to have mass \textendash would be predicted as massless, like the photon. \cite{Fermi:Electroweak}

\subsection{CP-violation}
CP-violation is violation of \emph{Charge conjugation Parity symmetry} \textendash the statement that the laws of physics should be unchanged under the exchange of a particle with its antiparticle. and under the inversion of spatial coordinates. It was proposed after the discovery that the weak interaction breaks parity symmetry, in order to ``restore order'' in the Hilbert space of quantum mechanics. In 1964, clear evidence was presented that weak interactions violated not just C-symmetry or P-symmetry but also the combined CP-symmetry \cite{CroninFitch:CP}. ``Direct'' CP-violation arises mathematically from the Standard Model if there exist at least three generations of quarks.

Since CP-violation can be considered the ``true'' symmetry between particles and antiparticles, CP-violation \textendash that is, a violation of the symmetries of physical law acting on particles and antiparticles \textendash could explain the observed asymmetry between quantities of matter and antimatter in the visible universe (in fact, it is generally accepted as a prerequisite, or else the asymmetry between matter and antimatter must have been present at the universe's inception). Since the observed CP-violation in the quark sector of the Standard Model alone does not account for the matter-antimatter asymmetry, CP-violation is also being explored in the lepton sector; alternatively, CP-violation in the Higgs sector could make up the shortfall \cite{LHC:CP:Higgs}.

\subsection{Muon Physics}
Muons ($\mu^{-}$) and muon neutrinos ($\nu_{\mu}$) form the so-called ``second generation'' of leptons. Like the electron, the muon is a spin-$\sfrac{1}{2}$ elementary particle with an electric charge of -1. The muon is approximately 200 times more massive than the electron, weighing in at 105.7 MeV/c$^{2}$ (compared with the electron's 0.511 MeV/c$^{2}$). As a result of its higher mass, the synchrotron losses for muons are much smaller than for electrons.

Muons are a viable candidate for a new generation of lepton-lepton colliders due to their large mass relative to the electron \textendash not only do they suffer less from synchrotron energy losses but the centre-of-mass-energy of the collisions is comparatively higher due to this high rest mass. Like electrons, collisions between muons are ``clean'' as they are fundamental rather than composite particles and do not interact via the strong force. However,  the muon decays via the weak interaction with a mean lifetime of 2.2 \textmu s to an electron and two neutrinos (an electron antineutrino and a muon neutrino, to conserve lepton number). The relatively short lifetime of the muon compared with the electron puts constraints on the ease with which beams of muons can be created, cooled, and collided before the particles decay.

\subsection{Beyond the Standard Model}
\subsubsection{Supersymmetry (SUSY)}
SUSY is a proposed extension of the Standard Model which relates bosons to fermions \textendash  according to SUSY, each particle from the one group has a supersymmetric ‘partner' particle with a spin differing by a half-integer. If SUSY were unbroken, the partners would have identical mass and internal quantum numbers aside from spin; but since no partner particles have yet been discovered for SUSY to be correct it must be a `spontaneously broken symmetry' \textendash that is, the as-yet unobserved `superpartners', if they exist, must be more massive than their `ordinary' counterparts \cite{CERN:Supersymmetry}. There is some hope that these supersymmetric particles will be observed at the Large Hadron Collider when it starts operating in the 7-14 TeV range; but the observation of a rare decay which would have been even less likely under supersymmetry casts some doubt on the validity of the theory \cite{BBC:SUSY}. According to UCL's Dr Waters, ``it would be surprising if nothing showed up in the 7-14 TeV range, although it may not be SUSY.'' [\ref{interview:waters}]

\subsubsection{Gravity, String Theory, Extra Dimensions}
The Standard Model is generally considered incompatible with general relativity. One proposed extension of the Standard Model is the graviton, a hypothetical elementary particle which mediates the force of gravitation \textendash in the same way that the W and Z bosons mediate the weak force, gluons mediate the strong force, and photons mediate electromagnetism. As the source of gravitation is the stress-energy tensor, a second-rank tensor, it follows that the graviton must be a spin-2 boson. Like the photon it would have to be massless if it exists, as the gravitational field appears, like electromagnetism, to have unlimited range \textendash i.e. to be a Yukawa potential (wherein the magnitude of the potential falls off exponentially with the mass of the force-carrying particle, in this case a graviton) with m = 0:

\begin{equation*}
V_\text{Yukawa}(r)= -g^2\frac{e^{-kmr}}{r}
\end{equation*}

Gravitons are predicted by string theory, and may be detectable at the Large Hadron Collider by their absence: according to CERN, ``Collisions in particle accelerators always create balanced events \textendash just like fireworks \textendash with particles flying out in all directions. A graviton might escape our detectors, leaving an empty zone that we notice as an imbalance in momentum and energy in the event.'' \cite{CERN:Gravitons:Online,deAquino:Gravitons} It is also typical of extensions of the Standard Model including extra dimensions, such as string theory, to include new particles such as heavy spin-1 particles which could be detected at future particle colliders \cite{CLIC:Concept}.

\subsubsection{Leptoquarks}
Leptoquarks are hypothetical particles which exist in various extensions of the Standard Model and carry both lepton and baryon numbers \cite{PDG:Leptoquark}. Since both quarks and leptons are grouped by pairs into the three generations of matter, the existence of leptoquarks might point to a fundamental symmetry underlying the Standard Model. If leptoquarks exist, they could be created at hadron colliders like the LHC, or at lepton-hadron colliders such as the proposed Large Hadron-electron Collider (LHeC) extension of the LHC [\ref{interview:waters}]. Leptons would couple to quarks \textendash allowing the violation of conservations of baryon and lepton number (but preserving the difference between baryon and lepton numbers). Between 2003 and 2007 the ZEUS Collaboration searched for first-generation leptoquarks at HERA; however ``no evidence for any leptoquark signal was found,'' \cite{ZEUS:Leptoquark} and according to UCL's Professor Butterworth, ``any physics case based on beyond-the-Standard-Model physics for [a hadron-lepton collider] is nonsense.'' [\ref{interview:butterworth}]
