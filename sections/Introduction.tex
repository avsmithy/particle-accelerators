On the 4th of July 2012, the European Organisation for Nuclear Research (CERN) announced that the two teams of physicists at the Large Hadron Collider’s (LHC) ATLAS and CMS experiments have discovered a new subatomic particle with a mass of about 126 GeV, which was consistent with the predicted properties of the Higgs boson. The Higgs boson was named after Peter Higgs, who postulated in 1964 the existence of the Higgs field as an explanation for the origin of mass of fundamental particles in the Standard Model (SM) of particle physics. Based on this discovery, the Nobel Prize for Physics 2013 went to Peter Higgs and Francois Englert for the theoretical discovery of the Higgs mechanism, another discovery that served to cement the success of SM physics.
 
Since the discovery of the Higgs boson, research teams around the world have begun proposing what the next stage of particle physics research should be. The general consensus is that with the discovery, more questions and new fields have been opened up for discussion and research, and that these would require a new high energy collider to complement the LHC.
 
While the LHC served its purpose as a “discovery” machine, smashing protons at energies up to 14 TeV, proton-proton collisions produce a large amount of background events that result in “dirty” collisions, which are not ideal conditions for precise measurements of the newly-discovered Higgs boson. These measurements include mass determination, top Yukawa coupling and self-coupling of the Higgs – these measurements require high precision in cleaner collision environments, performed at different energy stages.
 
Besides studying the Higgs mechanism in greater detail, there are still unanswered questions within and beyond SM Physics that physicists hope the next stage in particle physics research can answer. These include gaining a greater insight into the constituents of dark matter and the abundance of matter in our Universe.
 
Hence, the general consensus within the physics community is that the next particle collider to be built needs to be a clean-collision collider (e.g. a lepton collider), capable of performing collisions at different energy stages in order to study different phenomena. Data collected from such a collider would then be the best complement to the experiments that have been run and will run at the LHC.
 
However, such proposals to build another billion-dollar particle collider are not without their critics. Some question the necessity of another particle collider right after the LHC, considering the size and cost of such a project when funding could be channeled to other areas of research in physics.