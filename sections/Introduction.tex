On the 4th of July 2012, the European Organisation for Nuclear Research (CERN) announced that the two teams of physicists at the Large Hadron Collider's (LHC) ATLAS and CMS experiments have discovered a new subatomic particle with a mass of about 126 GeV, which was consistent with the predicted properties of the Higgs boson. The Higgs boson was named after Peter Higgs, who postulated the ``Higgs'' field in 1964 as an explanation for the origin of mass of fundamental particles in the Standard Model of particle physics. Based on this description, the Nobel Prize for Physics 2013 went to Peter Higgs and Francois Englert for the theoretical discovery of the Higgs mechanism, another discovery that served to cement the success of Standard Model physics.
 
Since the discovery of the Higgs boson, research teams around the world have begun to discuss what the next stage of particle physics research should be. With the discovery, more questions and new fields have been opened up for discussion and research, and the general consensus is that the exploration of these fields will require a new high energy collider to complement the LHC.
 
While the LHC continues to serve its purpose as a “discovery” machine, smashing protons at energies up to 7 TeV, with an imminent upgrade to 14 TeV, proton-proton collisions produce a large amount of background events that result in “dirty” collisions, which are not ideal conditions for precise measurements of the newly-discovered Higgs boson, all of which include mass determination, top Yukawa coupling and self-coupling of the Higgs – these measurements require high precision in cleaner collision environments, performed at different energy stages.
 
Besides studying the Higgs mechanism in greater detail, there are still unanswered questions within and beyond Standard Model physics that physicists hope the next stage in particle physics research can answer, exploration of which might give us a greater insight into the constituents of dark matter and the abundance of matter in our Universe.
 
Hence, the general consensus within the physics community is that the next particle collider to be built needs to be a clean-collision collider (i.e. a lepton collider), capable of performing collisions at different energy stages in order to study different phenomena. Such a collider would then be the best complement to the experiments that have been run and will run at the LHC.
 
However, such proposals to build another billion-dollar particle collider are not without their critics. Some question the necessity of another particle collider right after the LHC, considering the size and cost of such a project when funding could be channeled to other areas of research in physics.