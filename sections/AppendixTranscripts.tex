\subsection{Interview Notes}
\subsubsection{Dr. David Waters, 4/2/2014. Mary O'Donnell \& Harapan Ong}
\label{interview:waters}
\paragraph{Mary O'Donnell's Notes}
\begin{itemize}
\item Maybe it's better to wait for ten years and see what the LHC produces.

\item There is much more consensus than there was three years ago. Now there is a Higgs and we know what its mass is; probably 70\% of particle physicists would say build the linear collider now.

\item LEP did amazing precision measurements of the W and Z bosons and their properties. ``These things go in cycles.''

\item There are some people who say forget about doing a lepton collider: we need to push back the energy frontier even further and see what we discover in new energy ranges.

\item The mass isn't enough to confirm that the Higgs observed at LHC is the basic Standard Model Higgs but a lot of its other properties will be confirmed at LHC given time. Although hadron collisions are `messier' than lepton collisions it is still possible to analyse the properties of the Higgs albeit to lower precisions than using a lepton collider.

\item CMS just published a paper showing that the Higgs couples to fermions — it was discovered through its couplings to bosons but we have observed it coupling to fermions. So eventually a lot of the Higgs properties will eventually be confirmed at the LHC.
Linear colliders may measure subtle deviations in properties of particles discovered at the LHC which support various extensions of the SM.

\item Higher luminosities provide access to rare interactions.

\item There's a lot of indirect evidence for SUSY. It would be surprising if nothing showed up with the LHC's increase in energy. “It's early days,” at the LHC.
\item ``We could build the ILC to do all the Higgs measurements but it would be very frustrating if the LHC uncovered a supersymmetric particle just outside the ILC's energy range and you can't see it, and you can't upgrade the collider.''

\item $\gamma\gamma$-colliders: ``My first thought is that I don't know too much about that;'' suggests that  designs will use beams of electrons but collisions will take place between virtual photon `cloud' which surrounds electron.
\end{itemize}

\paragraph{Harapan Ong's Notes}

\begin{itemize}
\item Things to consider when choosing a collider:

\begin{enumerate}
\item \emph{Scientific Reach}\\
When considering the potential scientific reach of a collider, there are four main parameters to consider. The first two (arguably most important) are the energy and the luminosity of the collider. The energy of course determines the type of particles that the collider is able to produce, and the luminosity of the collider determines the probability of observing rare processes (high luminosity means a higher chance of seeing rare processes happening). The third parameter is the beam quality, where a high beam quality means a small energy spread (we want monoenergetic beams if possible). Finally, the type of particle used is important, whether it is protons or electrons or muons.

\item \emph{Feasbility}\\
On the high end of feasibility is the ILC, which is obviously the most robust and mature of all the proposed colliders out there. The technology required for the construction and operation of the ILC has been demonstrated at least at a laboratory scale, and now some of them simply require scaling up to the industrial scale. On the low end of feasibility would be muon colliders. The main challenge lies in how to cool and accelerate these muons that are produced with a wide energy distribution. The Muon Ionisation Cooling Experiment (MICE) has been running for some time now, and have yet to produce (to Dr. Waters' knowledge) any definitive demonstrations of muon ionisation cooling.

\item \emph{Cost}\\
There are a few factors when it comes to considering the potential cost of a collider. For example, the ILC is considered to be pretty expensive. The power required for the collider is essentially electrical (a gross simplification would be a wall-power socket to power the collider), which would lead to a rather long collider for just 0.5 GeV, which would increase cost. This would also result in a considerable power consumption. In order to decrease the length for a given energy, you could increase the accelerating gradient of the collider, requiring less accelerating structures needed overall and hence shortening the collider (reducing overall cost).
\end{enumerate}

\item CLIC, as the name suggests, would be more compact as it runs on the two-beam acceleration system, which achieves the higher accelerating gradient required to shorten the collider. Hence it would require less accelerating structures, so technically it would be cheaper. But let's not forget that CLIC requires two tunnels for two beams... So is it necessarily cheaper? We don't know\textemdash only ILC has a really robust cost breakdown, as CLIC is less ready.

\item Muon colliders lack a cost structure too, but a neutrino factory (which is touted as the first phase of the muon collider) could very well have a cost structure. 

\item While the ILC is pretty much ready-to-go, CLIC might require another 10 years of R \& D to prove the two-beam acceleration system, while Muon Colliders would require more than 10 years of R \& D to demonstrate muon ionisation cooling.

\item Historical reasons: Previously, the W and Z bosons were found at a hadron collider. LEP was built (a lepton collider) and used to measure the properties of W and Z very successfully and precisely. Hence, we would want to repeat history\textemdash a hadron collider has found the Higgs, so maybe it's time to build a lepton collider.
\end{itemize}

\clearpage

\subsection{Interview Transcripts}
\begin{table}[!ht]
\begin{tabular}{@{}p{0mm}p{5mm}p{130mm}@{}}
& JB: & Prof. Jonathan Butterworth\\
& RT: & Prof. Robert Thorne\\
& MO: & Mary O'Donnell\\
& HO: & Harapan Ong
\end{tabular}
\end{table}

\subsubsection{Prof. Robert Thorne, 08/02/2014. Mary O'Donnell}
\label{interview:thorne}
\begin{table}[!ht]
\begin{tabular}{@{}p{0mm}p{5mm}p{130mm}@{}}
& RT: & We're gaining a lot of data at 7 and 8 TeV centre of mass-energy. I think the collider has actually accumulated more data than it thought it would but it was initially going to run at higher energies as well, so it's been shut down\textemdash it will be turned on again next year at 13 and 14 TeV. And they found the Higgs boson, of course, as we heard but that was in many respects expected. It wasn't guaranteed that they would but it was expected or hoped that there would be signs of other things, and there really is no sign whatsoever\textemdash\\\\

& MO: & Like supersymmetric particles?\\\\

& RT: & Yes, no sign at all of supersymmetric particles which means that the lowest energy ranges of finding supersymmetric have been ruled out\textemdash and in some senses that was the most natural ranges to find them because part of the reasons for supersymmetry is to explain the mass differences between particles and the most natural explanation would be if the supersymmetric particles themselves would have masses similar to the Higgs and the W and the Z.\\\\

& MO: & Right, because I think I read about SUSY that if SUSY was perfect [the supersymmetric particles] would have the same mass as their `normal' particles\textemdash\\\\

& RT: & Yes.
\end{tabular}
\end{table}

\clearpage

\begin{table}[!ht]
\begin{tabular}{@{}p{0mm}p{5mm}p{120mm}@{}}

& MO: & \textemdash so we're expecting SUSY to be broken so they're somewhat more massive but I suppose it would make more sense if they were closer...\\\\

& RT: & Yes, it's more natural. The models of SUSY that people thought were most attractive were the ones that had the masses relatively light in general such that we would have found them already. But you can just change\textemdash I mean there's a lot of free parameters in any theory of SUSY so you can tweak the parameters a little and make the masses go up. But I think most people would agree it becomes less and less attractive the higher you push the energies; so there are some people who would say ``We didn't find them at 7 and 8 TeV\textemdash that's okay\textemdash but if we didn't find them at 13 or 14 TeV then it's really starting to look doubtful.'' Other people would say they may be just out of the range there. But I don't work on SUSY\textemdash I was perfectly happy to think that it might exist before the LHC turned on. I thought it was a perfectly attractive theory. The fact that there hasn't been any sign of it at all, personally, I'm rather more sceptical about it now. I would have thought there would have been some sign already. But that doesn't rule it out.\\\\

& MO: & Yeah. That pretty much covers my questions about SUSY. I also read that some extensions of the SM with extra dimensions also predict new particles which might be discovered in a collider.\\\\

& RT: & As far as I understand it those are rather more speculative theories and there's rather less to tie you down to a particular energy scale in those cases. So it could be a few TeV, it could be a thousand times more, a million times more, or whatever. I don't think there's quite the same argument, but then there's not quite the same arguments for SUSY either. The extra dimensions one actually appeared during my career and it always seemed slightly more contrived to me.\\\\

& MO: & Do you think then that the next collider should stay within the energy range\textemdash to be like a Higgs factory and to stay within the energy range to examine the Higgs, or would it be better to wait for a few years and see what comes up at the LHC and then build something a little bit more like CLIC?
\end{tabular}
\end{table}

\clearpage

\begin{table}[!ht]
\begin{tabular}{@{}p{0mm}p{5mm}p{120mm}@{}}
& RT: & Somebody may decide to go ahead with a not particularly high energy and perhaps not the best possible linear collider, which would be, like you say, a ``Higgs factory''. You'd just have electron-positron collisions or perhaps muon-antimuon collisions at the right energy to produce Higgs, which would be much more precise, or measure some things much more precisely than you can at the LHC, because you just have clean initial states. Electrons are fundamental particles whereas protons are made out of a mess of quarks and gluons. So there's always a limit in precision that you can get from that. But I think any really major development, like an extremely high luminosity and possibly higher-energy linear collider, or the Very Large Hadron Collider, which is being talked about at the moment\textemdash I think before anyone commits to anything they'll want to see some evidence of what's coming out of the higher running of the LHC. I have to be honest, I'm not sure which is the better way to go at the moment. If we don't see anything at all at some point looking into Higgs production at real precision is the only collider-based way that we're going to get evidence about new [physics]\textemdash if moving the energy up sequentially isn't getting us anywhere at all that would be the thing to focus on.\\\\

& MO: & I read that they want to look at CP-violation in the Higgs sector, and I read that that could be a possible explanation for the baryonic asymmetry in the universe.\\\\

& RT: & I have to be honest: I'm not sure how plausible CP-violation in the Higgs sector is as a model for that. CP-violation is necessary and the only form of CP-violation we have in the SM is in the quark mixing, and the neutrinos now that we've discovered the neutrinos have masses. But again, that requires some additional neutrino physics beyond that. People argue that neutrino masses is beyond the SM physics. I always think that's a slightly odd way of putting it because we chose the neutrino masses to be zero in the SM, you can give them masses, and that's it. That's all you have to do: you can give them masses and that's the new SM of neutrino masses.
\end{tabular}
\end{table}

\clearpage

\begin{table}[!ht]
\begin{tabular}{@{}p{0mm}p{5mm}p{120mm}@{}}
& RT: & However, because the masses are so small there's more of an indication that the mechanism for them getting their masses must be complicated and must involve beyond the SM physics, and there are ways there that you can get large instances of CP-violation and the same potentially with the Higgs. But, I think\textemdash the natural amounts of CP-violation we tend to get all seem to be much smaller, certainly ones that we've measured in the SM so far are enormously smaller than the ones required for the baryon asymmetry. But certainly looking at Higgs production a more precise machine would certainly be better than more energy ones.\\\\

& MO: & Dr. Waters told us that a lot of the measurements they want to make of the Higgs at a lepton collider could also be made at the LHC but it would take longer and would be to a lower degree of precision.\\\\

& RT: & You ultimately have some limit in precision at a hadron collider that, no matter how long you run, you're not going to get beyond because QCD is just a messier, less precise theory than electroweak theory just because the coupling is stronger. And that results in both the theory calculations being a lot harder to produce the same sort of precision and there are limits in how good they can get which are on the order of a few per cent or a couple of per cent.\\\\

& MO: & It can only be solved numerically, is that right?\\\\

& RT: & No, a lot of the calculations in QCD are based on the same idea as QED, which is that you have this coupling constant and you do an expansion in powers of the coupling constant\textemdash and in QED the coupling constant is on the order of $\frac{1}{100}$ or something, so the first term is very accurate and the next term is on the order of $\frac{1}{100}$ to that and the next term is $\frac{1}{100}$ to that and that's why some things have been calculated to about eight decimal places in QED because you calculate out to fourth order. It gets enormously more complicated as you go from one order to the next.
\end{tabular}
\end{table}

\clearpage

\begin{table}[!ht]
\begin{tabular}{@{}p{0mm}p{5mm}p{120mm}@{}}
& RT: & Whereas in QCD we've calculated out to third order in various things but the strength of the coupling is about 0.1 or 0.2, so your first correction has still got a 10-20\% error on it, and then it's $0.1^{2}$ or $0.2^{2}$, which is still a few per cent. That's about as good as we're about to get theoretically in the near future. But it's also that when you collide protons together nearly everything that comes out is hadrons and you just have so much background that getting extremely high precision for anything in that experimental background is\textemdash you think, again there are limits that are higher than the limits you've got in lepton-lepton collisions. You still produce QCD particles in the final state\textemdash you produce quarks from colliding leptons but you don't get as much background.\\\\

& MO: & You mentioned muon colliders a while back. I was going to ask you about the feasibility of that\textemdash because to me, they decay so quickly I just can't even conceive of it.\\\\

& RT: & Well, I'm a theoretical particle physicist and I'm working the QCD area so I don't really know very much at all. Muon colliders certainly do exist; they'd be linear colliders rather than the circular colliders because we have to create them, as you say, and then collide them rather than keep them circulating in a beam for a long time as they will just decay. It's more difficult to keep things circulating in a beam at high energy anyway because they give off photon radiation.\\\\

& MO: & Synchrotron radiation?\\\\

& RT: & Yes. So, I mean muon colliders\textemdash I'm afraid I don't really know that much.\\\\

& MO: & As far as I know, they suffer less from synchrotron radiation and they get up to higher energies but they're still really clean.\\\\

& RT: & Oh yes, in terms of the final state muon is identical to electron, basically. Colliding, as far as we're aware, and if there are any violations they're very small ones\textemdash the final states you get colliding muons are identical to the ones you get colliding electrons.
\end{tabular}
\end{table}

\clearpage

\begin{table}[!ht]
\begin{tabular}{@{}p{0mm}p{5mm}p{120mm}@{}}
& MO: & Do you know anything about $\gamma\gamma$-colliders? That was another thing that we don't seem to be able to find too much information about. But we keep reading that there are designs for $\gamma\gamma$-colliders that would be much more energy efficient and they could still examine the Higgs but at lower energies than a lepton collider would need.\\\\

& RT: & That comes from the same basis that the radiation you get off the other particles can be correlated into a beam and used as a collider in its own right. In a sense, you can measure the collisions of the practically-real photons that have been radiated off the proton, for example, with the proton. That's been not-so-much the case at the LHC but at the HERA collider they did have the measurement of cross-sections for collisions of the photons which had been radiated off the electrons with the protons. But I think there's quite a lot of development needed for that sort of thing.\\\\

& MO: & The only design I've seen has been involved two parallel linacs for electrons and they'd bring them together and I think the photons would collide but it would be a beam of electrons. I didn't fully understand how that would work.\\\\

& RO: & I don't think it's one of the primary next ideas. It seemed, I think, very unlikely not that long ago but they are talking very seriously about having the VLHC...\\\\

& MO: & I think I read that originally they want to do TLEP, which would be a very large lepton collider in the 80km ring, which would do precision measurements of the Higgs and then they would replace it with the VLHC in the same ring in much the same way they did the LEP collider and the LHC.\\\\

& RO: & That would seem like an obvious progression. The big problem there is the cost.\\\\

& MO: & Having to dig the tunnel.\\\\

& RO: & Yeah. I mean the simple digging of the tunnel is a very large amount of the total cost of the project.
\end{tabular}
\end{table}

\clearpage

\begin{table}[!ht]
\begin{tabular}{@{}p{0mm}p{5mm}p{120mm}@{}}
& MO: & Um, yes. I also read about the Superconducting Super Collider, which was also going to be about 80km but of course that got abandoned by the US government. And that was because the cost of digging the tunnel was prohibitively expensive. And politics as well.\\\\

& RT: & They dug quite a lot of the tunnel.\\\\

& MO: & Like 20km of it, yeah. And they're also competing with the ILC and CLIC. I think that TLEP could get up to higher energies than CLIC\footnotemark. I'm not 100\% sure about that though.\\\\

& MO: & So, how soon do you think a new particle collider needs to start being built in order to capitalise on the physics that's been done at the LHC?\\\\

& RT: & Well, the physics at the LHC is intended to go on for something like another twenty years or so. However, the LHC took at least twenty years from it being more-or-less decided that it was going to happen to it turning on. So, certainly for anything beyond a lepton collider, which isn't far beyond anything we've already done before, is going to take a couple of decades probably from start to finish. So I think that if it's going to follow on relatively soon from the LHC it is going to have to start development pretty quickly. But, as I said, if I had any role in the decision-making I would be extremely keen to see at least the first year or two of the 13-14 TeV running before we commit to anything.\\\\

& MO: & Just to see if anything else comes up?\\\\

& RT: & That could really make it quite clear that some things are preferable to others.\\\\

& &

\textemdash\\

& &

\footnotesize{$^1$ This is incorrect, actually. TLEP in fact has a proposed centre of mass-energy of 350 GeV, compared to a lower bound of 500 GeV for ILC and 1 TeV for CLIC.}
\end{tabular}
\end{table}

\clearpage

\begin{table}[!ht]
\begin{tabular}{@{}p{0mm}p{5mm}p{120mm}@{}}
& MO: & Right, because obviously if we don't find anything in the higher energy range it just seems like a huge waste of R \& D to look into building something like CLIC, which requires so much more demonstration of feasibility. But then if we do discover something in the higher energy range it just seems ridiculous to build the ILC when we can't examine those particles.\\\\

& RT: & Well, a variety of different things can happen. If you just discover a few hints of something and it's clear that you would discover more at higher energy then the VLHC starts becoming a very attractive prospect if the funding can be found. But if you don't find anything at all then there's no reason, if there's nothing at 14 TeV I don't think anyone's going to come up with any particularly overwhelming reason why there's going to be something at 40 TeV instead. There may be but I don't think there would be particularly strong arguments to suggest that there will, in which case looking at really high precision would seem like the obvious thing to do. It's quite strange though to think that whichever one it will be it's not going to be turning on until roughly when I'm retiring.\\\\

& MO: & Yes! Because it's not going to be for about twenty years. It's interesting, if I stay in physics that will be within my career. I'll be able to watch that happening. The other thing that Dr. Waters mentioned was the possibility that within an electron-proton collider like the LHeC they might discover theoretical particles\textemdash leptoquarks?\\\\

& RT: & This is something they were looking for at HERA and at one point got excited because there were an excess of events at high scales. Yeah, the LHeC is a slightly different project where you would be putting another beam into the existing LHC, that you have a very large energy version of HERA. That would be\textemdash not cheap but it's only essentially a modification to the LHC rather than an entirely new collider. In a sense, the sort of things that I've worked on, which are QCD and the structure of the proton and so forth, that would be extremely attractive and in some senses if you're continuing in the field of hadron collider physics it would be useful in its own right because it would provide more information on protons.
\end{tabular}
\end{table}

\clearpage

\begin{table}[!ht]
\begin{tabular}{@{}p{0mm}p{5mm}p{120mm}@{}}
& RT: & You get cleaner information on protons by colliding them with electrons because you've only got one 'messy' bit of the incoming state rather than two to worry about. So you can get rather more accurate measurements and therefore you would learn more about things like quarks and gluons within the proton from that regardless of what you might learn about beyond-SM physics. And that in itself might be useful for the future.\\\\

& MO: & In analysing hadron collisions?\\\\

& RT: & In analysing hadron collisions because it would provide more precision in that area because you would know more about the proton. It would be the ideal way to look for these things called leptoquarks, which are things which carry both lepton and baryon number. They are not one of the most standard BSM models. They're not supersymmetry. I think that's one of the reasons why this has had a bit of difficulty gaining any real traction. Not many people come up with particularly strong arguments for why there should be leptoquarks.\\\\

& MO: & The most compelling reason I read was that basically, ``It would be nice.'' Because it might explain why there's this symmetry [between leptons and quarks]. But I couldn't see any particularly compelling argument beyond that rather than that it would be nice.\\\\

& RT: & The vast majority of BSM do not automatically include leptoquarks is, I think the fair way to put it. You can construct ones which do but they're ones that might be there rather than that there are any good reasons for them to be there. People argue that there are a number of reasons why SUSY has good reasons for being there\textemdash they're reasons rather than proofs. So they're reasons why it would be nice and why it would solve some problems of the current SM. I'm not particularly aware of there being similarly strong arguments for leptoquarks so while you can't rule out that they're not there until you've done the experiment not that many people are expecting them to be there.\\\\

& MO: & Is there anything else that you want to add to that, that you can think of?
\end{tabular}
\end{table}

\clearpage

\begin{table}[!ht]
\begin{tabular}{@{}p{0mm}p{5mm}p{120mm}@{}}
& RT: & I think I've mentioned everything. I genuinely find it surprising at the moment that people are now talking seriously about the VLHC. A few years ago this wasn't something you heard about, but I think perhaps it is just the case that it's a surprise that the relatively large increase in energy\textemdash it's only a factor of 3-4 from the Tevatron so far, it will be 7 or 8\textemdash has not lead to anything at all, particularly unexpected.\\\\

& MO: & Although the Higgs was only just out of the reach of LEP so it seems that\textemdash\\\\

& RT: & I mean, going back twenty years, if you wanted to do something to discover the Higgs boson as quickly as possible then this wouldn't have been the way to do it. But on the other hand we wouldn't know as much about the absence of BSM physics as we do now, had we just built a slightly bigger lepton-proton collider.\\\\

& RT: & Ultimately we could have got there with the Tevatron, it's got corroborative evidence for the Higgs that, with the benefit of hindsight, knowing that it's there, is perfectly convincing but on its own it wouldn't have been a discovery. That would have got more and more convincing but\textemdash I'm not sure how plausible it would have been to have upped the energy at the Tevatron by a bit, which would have made the Higgs discovery more possible there. To have taken it to 2-3 TeV for example. It wouldn't have increased the rate very much but it wouldn't have needed to be very much. So, no, with the benefit of hindsight it wasn't the quickest way to have found the Higgs but if we'd have gone the route of the quickest way to find the Higgs we'd have even less idea about\textemdash we'd still have an absence of BSM physics but we might still be thinking that it was going to appear with a 7 TeV collider.\\\\

& MO: & All right! So, I think that's it, then.
\end{tabular}
\end{table}

\clearpage

\subsubsection{Prof. Jon Butterworth, 17/02/2014. Mary O'Donnell \& Harapan Ong} 
\label{interview:butterworth}
\begin{table}[!ht]
\begin{tabular}{@{}p{0mm}p{5mm}p{120mm}@{}}
& MO: & Okay, so, I mean. The main [colliders] we've been looking at right now are the International Linear Collider, and CLIC, but we've also been looking at TLEP and we were wondering if you had any thoughts about which one you think would be best.\\\\

& JB: & That's interesting. I think there are a lot of ways of answering the question ``what's best''. There's what's achievable and when, and there's what's the best science if you had all the technology and the money in the world\textemdash what would you do? I think that the ILC is the one that could be quickest, because it's the only one that could be built while the Large Hadron Collider's still running. Basically, the technology is ready for it, but it depends on the Japanese funding situation\textemdash it has to be them that do it. So, I would love to see the ILC built. I would love to see it producing Higgs bosons on resonance, which you can do\textemdash well, it's not quite a resonance but it's a threshold. It's $e^{+}e^{-}$ goes to a virtual Z boson which decays to a Higgs and a Z, and then by tagging the Z you don't care what the Higgs decays to, you can measure all the decays of the Higgs. So, you can do stuff with that that you can't do at the LHC, because the LHC is colliding protons, which are much more difficult\textemdash\\\\

& MO: & Because of the strong force, and it's a composite particle.\\\\

& JB: & Yeah, that's right. Basically there's no exclusive channel so you can't use conservation of energy to prove that there was something else there: in an $e^{+}e^{-}$ machine you know that the energy is what you put in, whereas in a hadron machine you don't know what the energy of the colliding quarks and gluons were, you only know what the energy of the proton was. So, we can use transverse energy balance because we know that transverse to the beam line there has to be a momentum balance, so we can get some information but it's not as clean as you could do at the ILC.
\end{tabular}
\end{table}

\clearpage

\begin{table}[!ht]
\begin{tabular}{@{}p{0mm}p{5mm}p{120mm}@{}}
& JB: & So, that's really good and I would also like to see them make measuremeants at the top-antitop threshold, so $e^{+}e^{-}$ goes to $t\overline{t}$, which happens at slightly higher energies\textemdash 350 GeV or so. The first one would be at the mass of the Higgs plus the mass of the Z, which is about 215 GeV. In fact, you need to go a bit higher than that, so it's more like 250 GeV. The second one would be at 350 GeV. So there's a programme where you could imagine building it in stages and going up in energy like that, and that would be brilliant if that was happening. It could be happening in parallel with the upgrades of the LHC, in just about ten years time if you're really lucky you could get something running\textemdash probably a bit longer than that but even so.\\\\

& MO: & But what if there was something else found at the LHC in the higher energy range? Because, I mean, we're going up to 14 TeV soon.\\\\

& JB: & Right, so I think the real thing here\textemdash we're talking about colliders now, so I'm leaving all the neutrino stuff aside, which is also very interesting but I'm going to leave that out for now. Talking about colliders, I think in the end the name of the game is as much energy as possible. I think that we have no guarantee now that there's anything there. So, we knew with the Higgs, we knew we'd either find it or prove it didn't exist, which is a really nice situation to be in. There is no killer case like that for the next one. So we could get an amazing surprise in the next run of the LHC, or in one of its upgrade runs later. We could find dark matter, we could find supersymmetry, we could find all kinds of stuff. But we might find nothing, or what we will find\textemdash it's not really nothing, what we'll do is we'll be showing that the standard model works in this new energy regime, which is important for electroweak symmetry breaking, which is where mass originates and where the Higgs is. So I think that's a great exploration to do, and if you're going exploring you want enough energy to get as far as possible basically. So then, that brings the whole TLEP and what you do at CERN into question.
\end{tabular}
\end{table}

\clearpage

\begin{table}[!ht]
\begin{tabular}{@{}p{0mm}p{5mm}p{120mm}@{}}

& JB: & So, whether or not the linear collider is built I think we want to try and put a case for a 100km tunnel at CERN. I think this is fantastic, I think CERN is the only place in the world that can deliver that, and it would just be a wonderful exploration. It would continue this exploration at high energies.\\\\

& MO: & Right, it seems like the biggest argument for TLEP is not the original collider itself, that would be a lepton collider, but the Very Large Hadron Collider, which would go in the same tunnel, because TLEP is actually lower energy than the ILC even.\\\\

& JB: & That's right. So, TLEP can only go up to\textemdash I don't think it can even get up to the top threshold, I think it can do the Higgs thing. I can't remember, but I don't think it can quite get to 350 GeV to do the top\textemdash\\\\

& MO: & I think I read something\textemdash I don't know, every time I say this it's always completely wrong, I look it up afterwards and I'm like, ``uh, no, that's not right,'' but I think it was something like 250 GeV.\footnotemark\\\\

& JB: & That would be the Higgs-Z threshold\textemdash so, yeah. I find it hard to believe that you've gone to all that effort of building that tunnel and then you put a low-energy beam in there first and wait another fifteen years or something for the hadrons. Certainly, if the ILC has been done there's no way you'll do TLEP, I think. It has some advantages, it gets a much higher luminosity than the\textemdash so you get more data, just because there's a storage ring, so you get more than one shot. You only get one shot with the ILC. So there would clearly be good physics to do that, but I find it very hard to imagine that you motivate digging an enormous tunnel and such an enormous project and then you say, ``well, we'll put the protons in in fifteen years time''. It's really hard to see. I mean, if we find something at the LHC in the next run that's unexpected, you can probably use the ILC energies to do a relaly good job of following that up.\\\\

& &

\textemdash\\

& &

\footnotesize{$^2$ This is indeed incorrect. TLEP has a proposed centre of mass-energy of 350 GeV, not 250 GeV.}
\end{tabular}
\end{table}

\clearpage

\begin{table}[!ht]
\begin{tabular}{@{}p{0mm}p{5mm}p{120mm}@{}}
& JB: & Then, you would do that I think and you would do it a lot quicker because you could clearly follow it up as much as you can with the LHC itself. Then I still think the next thing you want to do is another leap in energy, which means you would go to a hadron machine at whatever energy they can achieve in whatever tunnel they can build. So, to me, that's the most exciting thing and TLEP looks like a bit of a blind alley in that context. Certainly, if you built the ILC there's no call then to do TLEP. Well, no call is a bit strong\textemdash you could always do good science with it, because you'd have extra luminosity, so you'd get more precision on some things. But would you build that whole project just for more precision on some things? It doesn't sound like a killer science case for spending billions of dollars to me.\\\\

& MO: & What about CLIC, vs the ILC, given that CLIC can get up to higher energies but it's less\textemdash\\\\

& JB: & If CLIC can be made to work\textemdash the main difference between CLIC and ILC, apart from the energy reach of CLIC, is that we know that the ILC will work. We don't know whether CLIC will, and one of the main issues with CLIC, which I've not heard answered convincingly, is that the power budget is so high.\\\\

& MO: & Yeah, I read that. It's higher than even the entire power consumption of CERN while the LHC is on.\\\\

& JB: & And there are similar issues with TLEP as well, I think, because you're pumping energies in to make up for the synchrotron losses in TLEP, and in CLIC it's a different thing. But the thing with the LHC is, yeah, it's very expensive to build and the cryogenics\textemdash getting all that liquid helium together\textemdash and it does cost something in power but actually the superconducting magnet, once you've got the current going it's going, and you accelerate the beam and the beam just stays there and you collect data. It's not that power-intensive, actually. Whereas lepton beams, in a circular machine like TLEP, you're pumping energy in to make up for the synchrotron radiation losses. In a machine like CLIC or the ILC you're losing the beam every time, so you've got to start again from scratch.
\end{tabular}
\end{table}

\clearpage

\begin{table}[!ht]
\begin{tabular}{@{}p{0mm}p{5mm}p{120mm}@{}}
& MO: & Right, yeah, because you can only collide it one time because it's linear.\\\\

& JB: & There are things like energy recovery linear accelerators, which is an example I think at Daresbury, that they've tried this out, and you can not lose all the energy. You can do some things with it. But still there's an efficiency cost then. So, I think, my idea would be that we do ILC to do precision Higgs and top physics as soon as possible and then CERN comes back with a hadron machine\textemdash not TLEP.\\\\

& MO: & So you think that that would be more fruitful.\\\\

& JB: & I think so.\\\\

& MO: & I guess because the problem with supersymmetry is that, if it turns out to be correct you don't know what energy range it's going to come in. So you can't rule it out after going up the higher energy range at the LHC.\\\\

& JB: & At some level that's right, there's no negative. I think many people would say that if we see no sign of supersymmetry in the next run of the LHC then people won't give up on supersymmetry but they may start giving up, probably will start giving up on the idea that it has anything to do with the fine-tuning and naturalness problem\textemdash anything to do with electroweak symmetry-breaking. You can still put it in there but it means it's very unlikely it provides a dark matter candidate and it's very unlikely that it's got anything to do with the Higgs if we don't find it in the next run.\\\\

& HO: & Could I ask about\textemdash you said that the ILC is the most feasible, so it gets the fastest results. What's the penalty of not doing it fast enough? If I argue, say, we should let the LHC run for another ten years and see if anything comes up\textemdash why not wait? What's the penalty for waiting?\\\\

& JB: & There's a generational issue, right?\end{tabular}
\end{table}

\clearpage

\begin{table}[!ht]
\begin{tabular}{@{}p{0mm}p{5mm}p{120mm}@{}}
& JB: & If you lose the momentum in the field then you end up with a whole generation of physicists that all they've ever worked on is a machine that's coming to the end of its life, they've never seen anything built, they've never built anything. So you lose a lot of the technical skills, you lose a lot of the motivation. You should always remember these things are done by people, so if you've got a thirty-year career in the science and you don't see a single machine built during that career, who's going to go and do particle physics? We'll just stop doing particle physics in the end if you're not careful. The LHC will run itself into the ground and then we'll not have anything. So it's a sociological issue at some level; I would argue that you also lose a lot of the economic benefits of doing particle physics if you do that too because you're just not getting the technological push, you're not getting the people trained. So it's not just\textemdash\\\\

& MO: & You're not getting the spin-offs from it either.\\\\

& JB: & Yeah, that's right. You know, one of the great benefits of particle physics is that it trains people to go off and do other things as well and you're not giving people the right training if all you're doing is getting them to run a bit of code that someone wrote twenty years ago on a machine that was built thirty years ago.\\\\

& HO: & So it's not pushing, essentially\textemdash the idea of not waiting is to make sure there are always new things for people to do.\\\\

& JB: & Yeah, that's right. Now, whether those things need to be a huge collider or not\textemdash certainly to build a white elephant, to build a useless collider would be even worse because then you're training people to do something pointless. So there has to be a good science case, I'm not saying you just keep doing it for the sake for it to keep the field going so we can keep the field going. That's too circular an argument. But if you buy the idea that exploring the energy frontier is one of the drivers of fundamental science and fundamental science is one of of the drivers of the whole of research and of the economy in the end as well, then would you really want to let that way around another thirty years before you invest anything new in it? I would say, no.
\end{tabular}
\end{table}

\clearpage

\begin{table}[!ht]
\begin{tabular}{@{}p{0mm}p{5mm}p{120mm}@{}}
& JB: & We said already I'm not discussing neutrinos much here but there are other things you could do that would not be\textemdash rather than build the wrong collider it would be better to do these other things instead, so I'm not saying it's the only way and we must have it. I think we must be absolutely clear about our science case and there was an absolutely first-class science case for the LHC; I'm not sure there is one yet for all the other machines. I think there's a very strong one for the ILC given that we know where the Higgs is, but I'm not sure about the ILC at 2 TeV, for instance, or even 1 TeV. I'm very sure there's a brilliant science case for 250 GeV, where you do the Higgs plus Z, and for $t\overline{t}$ at 350 GeV. Beyond that? Depends what we find, don't really know. So, it's a fine balance between keeping the momentum going and having a future for the field if you think the field is of benefit, to have that exploration of the energy frontier. You've got to have future plans; on the other hand, to have the wrong future plan is even worse than having no future plan in some ways.\\\\

& MO: & What about maybe waiting a couple of years to see what the LHC throws out now that it's being upgraded?\\\\

& JB: & I think for the future of CERN\textemdash the other thing is, it's not a zero-sum game. So, ideally particle physics is a world activity. At the moment it's very European-led just because CERN has the highest-energy machine, so North America and Asia are all collaborating at CERN. They're getting in cheap at some level; the European taxpayer pays for the LHC mostly. They did contribute but not at the same level as the member states did, so they really ought to think about building the world-leading facilities. You might have seen in the news a couple of days ago that the UK has decided to join with the American project for the next big neutrino project, which is great news, so I would hope that if one could build an ILC either in China or Japan\textemdash most likely Japan but actually the Chinese do have other ideas as well\textemdash then that would be great and the area would be a world activity then. You could do it.
\end{tabular}
\end{table}

\clearpage

\begin{table}[t]
\begin{tabular}{@{}p{0mm}p{5mm}p{120mm}@{}}
& JB: & If Asia does not stump up a load of money to do that then I don't think Europe can do it on its own, and if you look at it from a Europe point of view then you have to look at, ``what do we do most efficiently and best at CERN with the infrastructure we have?'' and, to me, that's probably the big tunnel. Now, depending what we see it might be TLEP or the hadrons in there. I lean towards the hadrons first but you might want to do TLEP on the way. It kind of depends what we see. But I don't think there's any conceivable way that Europe, or CERN, can lead an ILC and then the big tunnel\textemdash it's just too much, it's too much technology, too much money, we don't have the people, either, to do it on our own. If this isn't a world activity it will just have to go slower. So, if the rest of the world can step up and do the ILC it will go faster and better.\\\\

& MO: & What about developing something like muon colliders?\\\\

& JB: & That's something we should do, but it's a long way in the future. I mean, it's kind of a backward agenda behind some of the neutrino experiments, actually, because the technology you need to make an intense neutrino beam is actually very similar to what you need to make an intense muon beam. And, in fact, if you had a muon storage ring you could get an intense neutrino beam from it as well. So, that fits very nicely with what they're doing in Fermilab. I mean, we say these things, you know, CERN does that, Fermilab does that, the Japanese do something or other\textemdash everyone's working with each other, it's just a matter of who steps up and really leads it. There are a lot of very interesting technologies being developed as part of neutrino experiments, and in the end I think they could lead to a muon beam but there are some big, big technical challenges to overcome. Are you aware of MICE for instance?\\\\

& HO: & \textemdash the ionisation cooling experiment.\\\\

& JB: & Yeah, yeah. So, no one really knows how to get a beam of muons up to the energy quick enough and close enough before they decay.\\\\

& HO: & Do you know how long they've been going for?
\end{tabular}
\end{table}

\clearpage

\begin{table}[!ht]
\begin{tabular}{@{}p{0mm}p{5mm}p{120mm}@{}}
& JB: & MICE? Yeah, they've been going for ages. It's not exactly a shining example of a successful project, I must say. It's taken a lot longer than they expected, and as far as I can see it's not really proven anything yet. So, also there's the target technologies\textemdash you have to get these muons from firing in a beam to a target somehow, and all of that's very challenging. The target technology's something you also need for neutrinos, and that is progressing and has progressed, but the cooling is something you really only need for an intense muon beam and it seems to be a bit of a blocker at the moment. So, I think it's a good idea: there are many advantages to colliding muons if you can, but there are also many challenges to overcome and I think in terms of technological readiness the ILC is the only thing we know how to build really. We know how to dig tunnels, so you can imagine\textemdash there's a lot of R \& D on high-field magnets so it's not unreasonable to think that this 100km tunnel, if you could get the resources, then you could do it. There's a lot of R \& D to do but you can see it would work in the end.\\\\

& MO: & I mean it does seem incredibly expensive, though, digging a huge tunnel\textemdash\\\\

& HO: & That's what killed the [Superconducting Super Collider] in Texas.\\\\

& MO: & Yeah.\\\\

& JB: & Well, it wasn't the cost of building the tunnel that killed the SSC. There are all kinds of complicated reasons why the SSC went down. It had massive cost overruns but they weren't particularly associated with the civil engineering. If you ask me, it failed basically for political reasons, not for technical reasons.\\\\

& MO: & Because the US weren't interested in particle physics.\\\\

& JB: & That's a bit strong! They weren't as interested as they needed to be. One of the stories that's told about it, may have an element of truth in it, is that the International Space Station and the SSC were seen as\textemdash you could only do one and they went for the space station. And then didn't fund the shuttle anyway, so.
\end{tabular}
\end{table}

\clearpage

\begin{table}[!ht]
\begin{tabular}{@{}p{0mm}p{5mm}p{120mm}@{}}
& MO: & Decommisioned them and didn't replace them.\\\\

& JB: & Yeah, that's right. And there was also, if you look at the publicity\textemdash I heard Christopher Llewelyn Smith, actually, who was the Director General of CERN when the LHC was approved, give his take on this which is, he did show real examples of communications about the SSC where on the one hand it'd be ``this is a huge national prestige project for the US'' and on the other hand saying ``this is a really important international collaboration'' and you can't have it both ways. You've got to either do it as a national project, where you have some helpers, which is what they've done with all their other big projects very effectively, or you do something like CERN where there's no one country in control. And they continually fell between those two goals, I think. The Japanese actually have to sort this out with the ILC as well and see\textemdash\\\\

& MO: & How much they want to contribute and control it.\\\\

& JB: & Yeah, and to be honest I think it's only going to really happen if the Japanese lead it and do it as a national project with international helpers. But there are plenty of people who think it should be a fully international project.\\\\

& MO: & I've actually wondered about locating a particle collider in Japan because of the number of earthquakes and other natural phenomena\textemdash\\\\

& JB: & There seems to be a tradition of building high-energy linear colliders in earthquake zones. The only other one that functioned was right across the San Andreas fault in California. Yeah, I think they could do it. I think that the site they have is through the mountain rock in the centre of the island, so it's pretty stable. But there's no guarantees anywhere, right?\\\\

& HO: & So, part of our review now\textemdash because our project is not constrained, we're not actually building the colliders\textemdash we can say that\textemdash
\end{tabular}
\end{table}

\clearpage

\begin{table}[!ht]
\begin{tabular}{@{}p{0mm}p{5mm}p{120mm}@{}}
& MO: & We want to pitch it for funding.\\\\

& HO: & But the thing is we have more freedom when it comes to choosing what we want. We can say, ``oh, ILC is too low-energy,'' as long as we argue it properly. You were saying there's no strong science case\textemdash I just want to know, what's the reason for building CLIC? What is, like, 3 TeV compared to, say, 1 TeV? What can we find, what are the proposed things?\\\\

& JB: & If you were someone who wanted a science case for CLIC\textemdash\\\\

& HO: & I'm a CLIC guy, let's say.\\\\

& JB: & Right, then, you know, the LHC when it turns on next finds a resonance at just under a TeV, say, or find new kinematic energies or signs of some new big deviation from the SM which is probably a new particle but we don't quite know what, somewhere between half a TeV and a TeV. Then the ILC will not be able to get that at all, chances are. We won't know if it's the lowest-energy SUSY state, maybe there's a lot more out there, or what is it? Especially if it decays to the electrons, for instance, $e^{+}e^{-}$, then that would be even nicer because you know it couples to them so you know you've got a chance of producing it at CLIC. So, but in general at the first look that there's a whole new spectrum of stuff out there people would immediately interpret it as a supersymmetry mass spectrum\textemdash new particles. If you're at the ILC, which is limited to a TeV eventually, and maybe would never get there, if you think CLIC can really do 2-3 TeV then that's really what you want, if you've got that kind of mass spectrum appearing at the LHC. If you don't find anything at the LHC, CLIC becomes really hard to motivate. ILC is still very well-motivated because the game then becomes pinning down precision parameters of the standard model, and the top couplings, and the Higgs width, and stuff like that.\\\\

& HO: & Those are very strong because we know it's there.\\\\

& JB: & Yeah, and that's a guaranteed science case that exists already for the ILC and it will still exist even if we find new stuff at the LHC.\end{tabular}
\end{table}

\clearpage

\begin{table}[!ht]
\begin{tabular}{@{}p{0mm}p{5mm}p{120mm}@{}}
& JB: & Those will be very important meaxurements to make and you'll still want them, but the question is, if we found new stuff at the LHC your priority will probably be to go and study that new stuff mor carefully rather than study the Higgs and the top\textemdash\\\\

& MO: & Well, and surely we could study that stuff at CLIC as well, if it was feasible and could be built.\\\\

& JB: & Yeah, you could but it would be a little like buying a sports car and then running it at thirty miles an hour. You'd be spending an awful lot\textemdash if you decided that's what you really wanted to do, you'd be spending an awful lot of money, and you probably wouldn't run it down at low energy. You'd probably go straight to the highest energy you can. But yeah, you'd do scans and in the end you could say, ``well, yeah, it could do that too,'' sure. It's all a matter of knowing there's only a certain number of particle physicists and a certain amount of money people are willing to spend on particle physics in the world.\\\\

& HO: & So you're saying at that high energy range that the next possible discovery would be, like you say, the super symmetry particles\textemdash?\\\\

& JB: & We don't know, right? The only thing that there's really experimental evidence for is dark matter, which may or may not be a supersymmetric state. We have experimental evidence that the standard model is not the whole story, and the only real experimental evidence for that is dark matter\textemdash the fact that we can see there's stuff out there and we can't see how to make it from standard model particles, so it must be something else. But we don't know its mass, so there's no guaranteed place for us to look for, to find it.\\\\

& HO: & It could be beyond 1 TeV, or\textemdash?\\\\

& JB: & It could be quite heavy. It could be way beyond the reach of what we've got, and people are looking for it in underground experiments as well, and astrophysical signatures are important as well, so it's an unknown. It's a big unknown.
\end{tabular}
\end{table}

\clearpage

\begin{table}[!ht]
\begin{tabular}{@{}p{0mm}p{5mm}p{120mm}@{}}
& JB: & But it certainly tells us that now we've found the Higgs we can't all pack up and go home because there's real experimental evidence, and there are many theoretical arguments for why the standard model is not complete as well. There's this fine-tuning of the Higgs mass, which you might have come across.\\\\

& MO: & Yeah, I read something about that but I can't remember off the top of my head\textemdash\\\\

& JB: & Basically, the quantum corrections to the Higgs, if you just treat it na\"\i vely, they're absolutely huge. So you have to end up with a cancellation of some several orders of magnitude, very precisely, to get 125 GeV. It looks unstable, from a certain point of view, if you try and make the standard model be our theory of everything all the way up to the Planck mass it looks kind of\textemdash it looks unstable, it looks like you continually have to re-tune the Higgs mass all the way up to keep\textemdash now that might just be that we're too stupid to understand the standard model properly. But it might be something, like\textemdash usually when there's a parameter like that, that's much lower than you think it should be, much nearer to zero than you think it should be, it's usually because there's an approximate symmetry in the theory, and in this case supersymmetry is the candidate approximate theory\textemdash one of the best candidates we have.\\\\

& MO: & So they get less and less approximate if you keep having to go to higher energies?\\\\

& JB: & Yeah, that's the thing. You have a supersymmetry breaking scale which is around the same mass as the electroweak breaking scale and then as you go to really high energies the symmetry's restored, and that cancels everything and keeps it neat, in very much the same way the Higgs breaks the electroweak symmetry and then at very high energies that symmetry's restored. So, you know, people put different amount of weight on that argument, depending on their taste. There's also\textemdash we don't understand why there are three generations of particles in the standard model, we don't understand why there's such a big matter-antimatter asymmetry in the universe\textemdash
\end{tabular}
\end{table}

\clearpage

\begin{table}[t]
\begin{tabular}{@{}p{0mm}p{5mm}p{120mm}@{}}
& MO: & I've read that that could be related to CP-violation in the Higgs sector, but then Professor Thorne told me that he didn't think that was very likely.\\\\

& JB: & I mean, the Higgs is a new thing so one of the things you look for in it is some kind of anomalous CP-violation but certainly if it's the standard model Higgs that's not the case. The neutrino sector, again, may have CP-violation in it but it's not clear if even that would help explain the whole thing. So, yeah, there's a lot of open questions. You want to\textemdash these are the questions we'd like to know the answers to, these are the technologies we can build, and these is the money we think we might be able to get to build them, and putting all that together is very tricky. At the moment, the upgrade, and the ILC, are the closest to being realisable financially and technically, and they both address some of those questions in a way that nothing else can at the moment. So, that's why they're important and good. And then after that, I don't know, I tend to think this tunnel is the only way to go. I'd change my mind if we see something, as the facts change.\\\\

& HO: & It seems to be that we alternate between a linear and a hadron collider.\\\\

& JB: & You mean electrons and hadrons?\\\\

& HO: & Yeah, yeah. At first they have a hadron collider, and then they build LEP, and then the LHC, and then\textemdash\\\\

& JB: & Yeah, it's true but I'm not sure that that's a pattern that I would extrapolate. It really depends that having both you learn a lot more. They're really complementary in several important ways. People say, well, but people used to say that hadron machines are discovery machines and lepton machines are precision measurement machines but then the gluon was discovered at an $e^{+}e^{-}$ machine and the most precise measurement of a W mass is from the Tevatron, which is a hadron machine. So it's not always that way. You've got to be very careful about looking too closely at the past and extrapolating it that way.
\end{tabular}
\end{table}

\clearpage

\begin{table}[!ht]
\begin{tabular}{@{}p{0mm}p{5mm}p{120mm}@{}}
& JB: & So, we will do a lot of precision physics with the LHC, that's the thing I'm most sure about. We will do a lot of precision studies of the Higgs there, and we'll do a lot of precision standard model cross-sections at the LHC.\\\\

& MO: & What about the LHeC?\\\\

& JB: & I think any physics case based on beyond-the-standard-model physics for that is nonsense. I don't think there's any motivation there for that. I do think there's a real strong motivation for understanding QCD better with it, and for doing high-density QCD and things, the structure of the proton, how quarks and gluons behave in a very dense environment, which I think is interesting. QCD is a fundamental force and it's the only one that is strong in the energy regimes we have access to. So, I'd like to see something like that done and you can imagine that if people do decide that we want to build some new lepton collider technology for one of the new machines you could maybe build a bit of it and collide it with the LHC. Maybe. But it seems, to me, like a very expensive project for quite a small area of science.\\\\

& MO: & Wouldn't it help us, if we learn more about the structure of the proton, that would help us with creating more precise hadron colliders, or analysing collisions?\\\\

& JB: & Not really, no.\\\\

& MO: & Really?\\\\

& JB: & I don't buy that argument at all. I know people put that forward and Robert [Thorne] might even have put that forward. But the problem is that we know an awful lot about the internal structure of the proton now from HERA, which is where I did my PhD, which was a brilliant experiment for that. You can measure a lot of it at the LHC as well, not all of it, you need the HERA data, but when you get to real super-high precision\textemdash to get better than HERA, if you look at what certainties are, and taking what we know about the proton from HERA and using it at the LHC, they're often nothing to do with the precision of the HERA data.
\end{tabular}
\end{table}

\clearpage

\begin{table}[t]
\begin{tabular}{@{}p{0mm}p{5mm}p{120mm}@{}}
& JB: & That's so precise that, that's great, that's not the limiting factor. The limiting factor is the theoretical uncertainties from taking something from electron-proton collisions and using it in proton-proton. And there's a theoretical model for the uncertainties that don't really get better by improving the precision of the HERA data. You can improve some of them by doing measurements at the LHC but it's not really clear to me that the LHeC would really benefit that at all. It's certainly very marginal, it's not a motivation for building the machine.\\\\

& HO: & Do you have any more questions?\\\\

& MO: & Um. Let me see if there's anything I wrote down that we haven't asked\textemdash I think that's about it, unless you had anything else to add that you think is important?\\\\

& JB: & Well, I've already mentioned that we need to diversify. Well, I haven't said this explicitly: I've mentioned the neutrinos. I think the whole of particle physics came together to build the LHC in a way, and huge\textemdash something like 75\% of the particle physics budget in the UK and probably in Europe is being spent on that one machine. We should not leap into the next one and assume that we will have one experiment that dominates the budget. You've got to be very careful, people have to remember how amazingly good the case for the LHC was. So, it may be that you build one of these things with the infrastructure we have at CERN but you do it more slowly and it's therefore, averaging over time, it takes a smaller fraction of the budget and that allows you to do more in the neutrino sector, more precision experiments, maybe dark matter searches. I think we need to allow ourselves a bit of time to digest what we're learning from the LHC and think very carefully about the next thing. I would say that if the ILC becomes a big Japanese project\textemdash that's brilliant, no qualms about that at all, it's a fantastic machine. But before we commit the rest of Europe to a new machine, in Europe I think we really need to have a bit of time just thinking and diversifying a bit.\end{tabular}
\end{table}

\clearpage

\begin{table}[!ht]
\begin{tabular}{@{}p{0mm}p{5mm}p{120mm}@{}}
& JB: & And that's what we're doing in the group here [at UCL]\textemdash trying to start little, SuperNEMO and a bunch of small-scale but important projects, and I think the world particle physics community needs to remember that you can do some great science in those small things as well. And then when we agree what the next big project should be and what the big questions it can address are then we go for it again and we all get together and do it again. And, I think it' a very exciting time at the moment but I think we have to be careful not to all jump on a bandwagon too quickly.\\\\

& HO: & I just thought of one more: I was reading something about $\gamma\gamma$-colliders. I don't know anything about it\textemdash it's mainly, when I read papers they seem to say that it's not great as a stand-alone collider.\\\\

& JB: & Yeah, they do it by firing lasers into an $e^{+}e^{-}$ collider usually, recoil photons off those.\\\\

& HO: & Some say it's a great add-on to your linear colliders, something like that. But apparently the ILC is not focusing on that.\\\\


& JB: & No, ILC is\textemdash as with a lot of these things, it's a bit like LHeC, it looks like a great add-on once you've got this thing going already and then you look into the details and it costs a lot of money, it's a lot of effort, and it's already very challenging to make ILC work. Then people put it lower priority and you never get as far as your second priority because you're so busy trying to make the thing work. So I think there's a lot of scepticism about it in that context. One of the nice things it would do, like we've measured Higgs to $\gamma\gamma$, like it would measure $\gamma\gamma$ to Higgs and we could measure its full width and everything like that. There are some things it can do better than $e^{+}e^{-}$ but if you had to choose between $e^{+}e^{-}$ and $\gamma\gamma$ you would definitely choose $e^{+}e^{-}$. Then it's a question of, can you actually afford to do something else as well? And I think most people at the moment think now, but I think if you ended up with a $e^{+}e^{-}$ machine, say you had the ILC that could go up to 350 GeV and then you were faced with a choice: do we want to upgrade this to a TeV, or do we want to make it a $\gamma\gamma$ machine that would be a very interesting discussion and it would probably depend on what you'd learnt on the way there as to which you would do.
\end{tabular}
\end{table}

\clearpage

\begin{table}[!ht]
\begin{tabular}{@{}p{0mm}p{5mm}p{120mm}@{}}
& JB: & But I think initially you would do the $e^{+}e^{-}$ machine if you can, and you'll worry about energy upgrades and $\gamma\gamma$ things later.\\\\

& HO: & Okay! I think that's all we have.\\\\

& MO: & Okay, yeah. That was great. Thank you very much for letting us come and talk to you.\\\\

& JB: & It's a pleasure.
\end{tabular}
\end{table}
