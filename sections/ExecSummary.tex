The aim of the project was to recommend the best option for the next high energy particle collider with a view to acquiring hypothetical funding for its development. We fulfilled this aim by critically assessing a range of proposed colliders in terms of their physics potential, technical feasibility, technological spin-offs and development cost, while also taking into account the timescale and political aspects of making the project a reality.

The decision making process took into account the following main objectives:

\begin{itemize}
	\item The physics potential of the next collider
    \item The necessary energy reach
    \item Feasibility in terms of technology and time schedule.
    \item What new technology will be produced \textemdash \, spin\textendash offs
    \item Development cost \textemdash \, funding models
    \item Location and governance
    \item Successes and failures of previous particle colliders
\end{itemize}

\subsection{Findings}

The project involved an extensive literature review. We began by evaluating a variety of proposed collider concepts including electron-positron (lepton) colliders, ILC, CLIC and TLeP, a photon collider, SAPPHiRE, the lepton-hadron collider LHeC and muon-muon colliders.

Additionally, a critical assessment of previous colliders was conducted from the 1987 Superconducting Super Collider, which served as a warning about the need for strong political support and adequate funding, to the LHC, which shows the success that can be achieved with a well-planned, well-funded project with a clear science case.

Following the collider evaluations we decided that rather than opt to push the energy barrier, the science community should build a collider that can precisely measure the properties \textendash mass, spin and interaction strengths \textendash of the recently discovered Higgs Boson. Understanding the properties will confirm whether or not this Higgs is that predicted by the Standard Model or instead a link to physics beyond the Standard Model. This in turn will influence the future of particle physics research.

The only way to achieve these measurements is by building a linear lepton collider. The collider must be linear so the energy reach is not be limited by synchrotron radiation. Colliding leptons result in clean collisions and enable us to probe specific energies due to the well-defined centre-of-mass energy.

As well as being the most robust collider proposals of those evaluated, the ILC and CLIC are both linear electron-positron colliders capable of accurately probing the Higgs. As a result we compared them in depth before coming to a well informed decision on the best collider to be built

CLIC presents a higher energy (3 TeV) and luminosity ($10^{35}$ $cm^{-2} s^{-1}$) than the ILC which will have a maximum energy of 1TeV and ten times smaller luminosity $\sim10^{34} cm^{-2} s^{-1}$. Both colliders will be capable of making precise measurements of the Higgs particle, top quark physics and potentially discovering the lightest supersymmetric (SUSY) particles. However CLIC is more viable for the production and observation of the lightest SUSY particles due to its higher energy and luminosity. While the ILC is ready to be built, requiring only industrial scale production of some components, the technology required for CLIC is not ready and requires a further 10 years of R\&D. Also CLIC demands higher running costs due to its 582 MW running power, compared to ILC's 200-250 MW running power requirement. CLIC would be built at CERN according to the conceptual design report and the ILC is likely to be built in Japan (following detailed analysis of potential sites).  

Both machines threw up similar spin-offs which benefit areas of medicine and industry. ILC is much more mature with detailed governance and funding models whereas CLIC is lacking in these respects.

We considered the need for a higher energy reach; accompanying the greater energy reach of CLIC is years of research and development which could be for nothing as the higher energy (14 TeV) LHC may find a particle that requires an even larger energy reach to be examined or indeed find nothing. Therefore why wait to spend 6 billion, building a higher energy machine that would be ran at lower energies. As Professor Butterworth said, ``this is like buying a sports car and only driving it at 30 mph.'' [\ref{interview:butterworth}]

Waiting for higher energy reach is risky and there is a case to be made for building soon \textendash a generation of scientists could go by without building a new collider; consequently there may be no driving force for new physics and a potent economic driver is also lost. On the other hand ILC is ready to go and could be built and running by 2026.

CERN could not cope with running both CLIC and LHC at the same time; it would make particle physics too Europe focused and the power budget of CLIC is too great to operate both simultaneously. In fact the problem of CLIC's enormous power budget has not been solved. The ILC is expected to be hosted by Japan and this provides a good opportunity to internationalise high energy physics through global cooperation.

\subsection{Conclusion}

We concluded that while CLIC would expand the energy frontier, this is not a sufficient reason to choose it over ILC. The science case is not strong enough as there have been no signs of new physics at the LHC and may not be at the higher energy. Therefore the optimal solution is to build the ILC which is more politically and technically feasible and has detailed governance and funding models. The ILC is well motivated to make important measurements like pinning down precision parameters of the Standard Model, the Higgs width and the top couplings; especially given that we know the Higgs is found at 250 GeV and the $t\bar{t}$  at 350 GeV. These measurements need to be made, regardless of whether anything exists at higher energies.

The ILC is well worth the \$7.8 billion development cost and is projected to have a financial impact of $\sim$\$40 billion over a 30 year period while also advancing the quest to explain the fundamental laws of nature and the universe.

Greater detail on the above objectives which lead to our decision to choose the ILC as the next High Energy particle collider is given in the main body of the report along with the accompanying appendices.