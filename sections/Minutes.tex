\subsection{Meeting One}

\begin{tabular}{l l}
  Date & 21st January 2014, 12:00 \\
  Location & Chadwick G01 \\
  Meeting Called by & Dr. Frank Deppisch \\
  Type of Meeting & Introductory \\
  Secretary & Mahera \& Esther \\
  Attendees & Izzy, Harapan, Tom, Josh, Mary, Mahera, Esther, Tim \\
  Absent & Alex (Interview) \\  
\end{tabular}

\subsubsection{First Half of Meeting}

\paragraph{Discussion}
\begin{itemize}
  \item Assessment is based on how well we work as a team, approaching problems quantitatively
  \item Why do we need high energy? It was thought that high energy is better because it produces more particles in collision. To produce a heavy particle high amounts of energy are needed. We must try to make particle density high to create more collisions.
  \item Energy and luminosity are two parameters that determine how we can use the collider.
  \item Proton-Proton collisions are dirty. Should we look into Electron-Proton collisions, as they are cleaner?
  \item We have only seen the Higgs, do we expect to see anything else?
  \item How expensive is it to keep the LHC running with the hope of finding something other than the Higgs?
  \item Does it make any sense to further the energy discoveries or should we create a new collider to probe the Higgs?
  \item People expected to see super symmetric particles but we haven’t so far so it has fallen out of favour.
  \item Every time the particle goes around the ring it is supposed to gain energy hence why they want to build bigger rings. The linear colliders have to travel up and then collide. Ring colliders have a higher luminosity and can constantly go around.
\end{itemize}

\paragraph{Conclusions}
Examples of colliders: Hera, Muon colliders, Electron-Proton colliders, Proton-Proton colliders, Tau particle colliders.

\begin{tabularx}{\textwidth}{X p{4.5cm} p{1.2cm}}
  \textbf{Action Items} & \textbf{Persons Responsible} & \textbf{Deadline} \\
  Look at UCL AGP  \textemdash \, particle physics group & Whole group  & N/A \\
  Look at physics of accelerating particles & Whole group  & 04/03/14 \\
  Interview Particle Physicist in UCL for introduction to physics behind the colliders & Mahera \& Esther if meeting is to be booked & 11/03/14  \\
\end{tabularx}

\subsubsection{Second Half of Meeting}

\paragraph{Discussion}
\begin{itemize}
 \item Would be helpful to research arguments for colliders e.g. scaled down for cancer radiotherapies  \textemdash \, Mary
 \item Look at future technology and the feasibility in 50 years?
 \item What are development costs?
 \item Research china building projects for Neutrino physics and how it has supported economic growths.
 \item Current economic climate, when should we build it?
 \item Pro: Projection of economic growth for areas with the colliders  \textemdash \, Japan
 \item Con: In current economic climate are there other more important things to think about?
 \item Think about timing, when should we run the new collider? We have two options, 1) Wait for the LHC to stop running, which may take too long, are they really going to find an other events? 2) Run at the same time as LHC, which may be too expensive.
 \item Include timeline or Gantt chart as to when to build it and when it finishes.
 \item Big part of the report is to compare different colliders, existing and exploration proposals.
 \item If we decide on an existing collider, we have to concentrate on why have we chosen this?
 \item If we decide on a new particle accelerator design, we have to concentrate more on the design rather than comparisons to other models
 \item Propose our collider to the particle physics community (Frank) and the governments of the world.
 \item Create our reports but maybe create a cool shiny flyer. Design a logo.
 \item LHC had a time frame of 50 years, so we assume we are looking at our collider to be built in 50 years.
 \item LHC, CLICK, TIP, ILC, VHLC, High luminosity (upgrades to the colliders)
 \item Plot energy vs. time chart for a 50-year time period.
 \item Not only to look at particle accelerators, but look at detectors.
\end{itemize}

\break
\begin{tabularx}{\textwidth}{X p{4.5cm} p{1.2cm}}
  \textbf{Action Items} & \textbf{Persons Responsible} & \textbf{Deadline} \\
  Confirm dates and office details with Dr Deppisch, send email to everyone & Mahera  & 21/01/14 \\
  Researching ILC  \textemdash \, International Linear Collider & Izzy, Alex, Tom  & \\
Researching TLAP & Harapan, Mary & 28/01/14 \\
Researching CLIC  \textemdash \, Compact Linear Collider & Tim, Josh & 28/01/14 \\
Researching history behind colliders & Esther, Mahera & 28/01/14 \\
\end{tabularx}

\subsection{Meeting Two}

\begin{tabular}{l l}
  Date & January 28th 2014, 12:00 \\
  Location & Chadwick G01 \\
  Meeting Called by & Harapan Ong \\
  Type of Meeting & Discussion \\
  Secretary & Mahera \& Esther \\
  Attendees & Harapan, Mahera, Esther, Josh, Alex, Tom, Mary, Tim, Izzy \\
  Absent & N/A \\  
\end{tabular}

\subsubsection{ILC - Tom, Alex \& Izzy}

\paragraph{Discussion}
\begin{itemize}
\item  The equipment which powers multiple-beam Klystrons has not been produced on an industrial scale yet
\item Will start at 0.5TeV but can be expanded to 1TeV
\item 2016 start, completion in 2026
\item Will use superconductors which will probably lose conductivity at up to 60K
\item Costs estimated at £7.8billion
\item 2 possible sites located in Japan

\item Further investigation into Higgs
\item Number, size and scale of extra dimensions
\item Other particles in Standard Model
\item Lighter supersymmetric particles
\item Disadvantages - Inefficient?
\end{itemize}

\paragraph{Conclusions}
Lepton collider whereas LHC is proton, therefore produces ‘cleaner’ collisions. 30\% proton polarization, can be expanded to 60\%.
Now that the energy range of Higgs is known, would be easier to find and reproduce.


\begin{tabularx}{\textwidth}{X p{4.5cm} p{1.2cm}}
  \textbf{Action Items} & \textbf{Persons Responsible} & \textbf{Deadline} \\
  Write pro/con list for ILC & Tom, Alex, Izzy & 30/01/14 \\
  Assemble list into presentation & Mahera, Esther & 04/03/14 \\
\end{tabularx}

\subsubsection{TLEP - Josh \& tim}

\paragraph{Discussion}
\begin{itemize}  
  \item  Design:
  \item Circular collider  \textemdash \, advances in magnet technology have made this more feasible
\item  Circumference of 80km
\item  Built around LHC
\item  Electrons and positrons
\item  Could be up by 2030 if started in 2020
\item  LHC could become accelerator for TLAP in the future: Proton-proton collisions at 100TeV
\item  Estimated at £7billion
Advantages:
\item  No. of observed Higgs per year will be high: 2106
\item  Greater luminosity for Z and top quark than CLIC
\item  Doesn’t require LHC to shut down
\item  Technology is largely already established: fewer unforeseen costs
\item  Looking at Z, W, top and Higgs
Disadvantages:
\item  Does not stimulate new design and technology
\item  Synchrotron radiation loss
\end{itemize}

\begin{tabularx}{\textwidth}{X p{4.5cm} p{1.2cm}}
  \textbf{Action Items} & \textbf{Persons Responsible} & \textbf{Deadline} \\
  Write pro/con list for TLEP & Tim, Josh & 30/01/14 \\
  Compile into presentation & Mahera, Esther & 04/01/14 \\
\end{tabularx}

\subsubsection{CLIC - Harapan \& Mary}

\paragraph{Discussion}
\begin{itemize}
 \item Design:
\item  Acceleration cavity at room temperature
\item  Two beams used: efficiency of 95\%, see PhD thesis.
\item  10km long
\item  Through LHC
\item  Two detectors like ILC
\item  3TeV
Advantages:
\item  Higher luminosity, shorter time
\item  At 3TeV, top quarks can be investigated
\item  Will have a greater impact on future technology
Disadvantages
\item  All the technology has not been produced yet, especially not at an industrial scale
\end{itemize}

\begin{tabularx}{\textwidth}{X p{4.5cm} p{1.2cm}}
  \textbf{Action Items} & \textbf{Persons Responsible} & \textbf{Deadline} \\
  Write pro/con list for CLIC & Harapan, Mary & 30/01/14 \\
  Assemble list into presentation & Mahera, Esther & 04/03/14 \\
\end{tabularx}

\subsubsection{Conclusion}

\begin{tabularx}{\textwidth}{X p{4.5cm} p{1.2cm}}
  \textbf{Action Items} & \textbf{Persons Responsible} & \textbf{Deadline} \\
  Ask group from last year for report? & Mary & N/A \\
  Review colliders: Cost, Feasibility, Spin-offs etc & Everyone & 04/03/14 \\
  Read up on gamma, plasma, muon colliders etc & Everyone & 04/03/14 \\
 
  Think about: if we design our own collider, what’s missing from other proposed colliders? & Everyone & 04/03/14 \\
\end{tabularx}

\subsection{Meeting Three}

\begin{tabular}{l l}
  Date & February 4th 2014, 12:00 \\
  Location & 203, 23 Gordon Square \\
  Meeting Called by & Dr. Frank Deppisch \\
  Type of Meeting & Discussion \\
  Secretary & Mahera \& Esther \\
  Attendees & Dr. Frank Deppisch, Harapan, Mahera, Esther, \\
  & Josh, Alex, Tom, Mary, Tim, Izzy \\
  Absent & N/A \\  
\end{tabular}

\subsubsection{Discuss Work Allocation \& Report Structure}

\begin{itemize}
\item 2-3 pages is enough for each collider
\item Include a table for comparison
\item Downfall of 2nd work allocation system is that it does not really compare colliders
\item Would produce disjointed report 
\end{itemize}

\paragraph{Conclusion}
 Choose both ideas: everyone write about 9 colliders, have others from separate sections compile information. We all write about a collider, even if just one page.
 

\begin{tabularx}{\textwidth}{X p{4.5cm} p{1.2cm}}
  \textbf{Action Items} & \textbf{Persons Responsible} & \textbf{Deadline} \\
Write 2-3 pages about chosen collider & Harapan, Tom, Tim, Izzy, Mahera, Josh
& 11/02/2014 \\
Write about basic collider physics and particle physics
& Mary, Alex
 & 11/02/2014 \\
 Compare colliders &
Esther & 
18/02/2014 \\
\end{tabularx}


\subsubsection{Presentation}

\begin{itemize}
\item \textbf{ILC}
\item  Investigating lightest supersymmetric particles
\item  Acceleration gradient not as good as CLIC
\item  Technology mostly seems feasible except Marx Modulators
\item  “Ready to go”, location has been picked (Japan)
\item  Spin-offs: Medicine, PET, X-ray production, grid networking, probing nuclear waste using gamma rays
\item \textbf{CLIC}
\item  Low feasibility but higher energy
\item  CP violation investigation
\item  High gradient → compact. Could be cheaper? But same length as ILC
\item  Spin-offs similar to ILC but different acceleration techniques
\item  At least twice as expensive as LHC
\item \textbf{TLEP}
\item  80km ring around LHC
\item  Major drawback  \textemdash \, cannot be looked at in detail as design study has not been completed yet
\item  Better for if we are designing our own collider
\item  Digging its tunnel could be difficult and costly, almost half of £7billion budget
\item  Because of circular design we need to consider beamstrahlung
\item  Circular collider means we observe many more collisions, interactions and high luminosity
 
\item \textbf{Muon}
\item  Not ready or feasible yet
\item  Lepton and circular  \textemdash \, clean collisions without significant synchrotron radiation due to large mass of muons
\item  Can possibly reach beyond energy levels of CLIC ~6TeV
\item  Small power consumption
\end{itemize}

\subsubsection{Additional Comments}

\begin{itemize}
\item W-Z boson found at Hadron Collider first
\item  Now there is momentum to probe Higgs
\item  Possible that we should not rush? If we wait longer we can find higher energy particles to investigate
\item  A lot which could be discovered at ILC could also be discovered by LHC
 
\item  General public support may not be necessary? However, waiting to build collider also has its risks
\item  Waiting for supersymmetric particles could be more interesting to public
 
\item  Building a collider could boost the GDP of the host country e.g. China
\item  Introduction to report needs to break down fundamentals of particle and collider physics, e.g. luminosity, energy, how to address technological problems, theory of energies at which we expect to find things
\item  Imagine it as an introduction to a lab report
\end{itemize}


\begin{tabularx}{\textwidth}{X p{4.5cm} p{1.2cm}}
  \textbf{Action Items} & \textbf{Persons Responsible} & \textbf{Deadline} \\
  Write pro/con list for CLIC & Harapan, Mary & 30/01/14 \\
  Assemble list into presentation & Mahera, Esther & 04/03/14 \\
\end{tabularx}



\subsubsection{Conclusion}

\begin{tabularx}{\textwidth}{X p{4.5cm} p{1.2cm}}
  \textbf{Action Items} & \textbf{Persons Responsible} & \textbf{Deadline} \\
  Everyone write 2-3 pages on their chosen collider/ physics topic
& Everyone
& 11/02/2014 \\
\end{tabularx}


\subsection{Meeting Four}

\begin{tabular}{l l}
  Date & January 28th 2014, 12:00 \\
  Location & Chadwick G01 \\
  Meeting Called by & Harapan Ong \\
  Type of Meeting & Discussion \\
  Secretary & Esther \\
  Attendees & Harapan, Mahera (Skype), Esther, Tom, Mary, Tim, \\ & Izzy,  Georgie Hoare \& Deborah Allan \\
  Absent & Josh \& Alex (Attending Group 10 Meeting) \\  
\end{tabular}

\subsubsection{Collider Reviews}
\begin{itemize}
\item  \textbf{ILC - TIM}
\item  Rising cost of £7.8billion  \textemdash \, in comparison LHC was £4-6billion
\item  ILC has accurate cost measurements in comparison to other possible colliders
\item  Biggest drawback is energy range
\item  Mary: We should wait
\item  Tim: In reality it may be hard to stop its momentum now but for the purposes of our report we can opt to wait
\item  \textbf{SAPPHiRE - Izzy}
\item  Cheaper than others, γ-γ Higg’s factory
\item  It could either stand alone in CERN or Fermi lab, or be combined with a linear collider
\item  Feasible as plans have already been drawn
\item  80GeV, lower energy
\item  Could investigate CP asymmetry violation
\item  Similar parameters to ILC e.g. luminosity
\item  Laser technology needed is not developed yet, council needs to be formed to oversee this
\item  Based on TLEP technology
\item  Thoughts:
1. Could we add a photon collider to ILC?
2. Would this cause the timeline of ILC to stretch significantly?
3. In terms of pitching to governments, investors etc, would it be safer to just go for ILC?
 
\item  \textbf{LHeC  \textemdash \, Tom}
\item  Hadron-Electron Collider
\item  No new technology needed
\item  More details on feasibility would be useful?
\item  Very low cost - ~£50million
\item  Even safer in terms of physics than ILC as we know QCD exists
\item  \textbf{TLEP  \textemdash \, Josh}
\item  Very large, 80km, so large engineering and tunnel costs
\item  Its main pro is the possibility of VLHC, and also very likely to be funded by CERN
\item  \textbf{Muon  \textemdash \, Harapan}
\item  Biggest con is cooling
\item  Neutrino physics would be very useful to help understand their mass, which is not predicted by the Standard Model
\item  Could be the next but one collider, e.g. after ILC or CLIC
\end{itemize}

\begin{tabularx}{\textwidth}{X p{4.5cm} p{1.2cm}}
  \textbf{Action Items} & \textbf{Persons Responsible} & \textbf{Deadline} \\
  Assemble table comparing all colliders &
Esther & 
18/02/14 \\
\end{tabularx}

\subsubsection{Structure \& Content of Report - Everyone}

\begin{itemize}
\item ILC Project Implementation Planning seems like a good template to use
\item  We can’t go very in depth in terms of Physics content
\item We should aim to a little higher than Nuclear and Particle Physics
\item Changes to plan, see Basecamp
\item Reports should be changed from bullet points to prose in order to fit the report
\end{itemize}

\begin{tabularx}{\textwidth}{X p{4.5cm} p{1.2cm}}
  \textbf{Action Items} & \textbf{Persons Responsible} & \textbf{Deadline} \\
Read changes to contents page: https://basecamp.com/2515911/projects/4773589-particle-accelerator/documents/4894157-contents-page &
Everyone & 
18/02/14\\

Change reports from bullet points to prose &
Everyone &
18/02/14 \\

Upload ILC Project Implementation Planning report to Basecamp &
Harapan &
12/02/14 \\

\end{tabularx}

\subsubsection{Choice of Collider - Everyone}

\begin{itemize}
\item Most choose CLIC, then second choice Muon  \textemdash \, Esther, Mary, Mahera, Izzy
\item Muon does not seem as feasible
\item We could possibly upgrade CLIC → muon, however this does not make sense as muon could use a much smaller circular collider and obtain higher energies than if using a huge linear collider like CLIC
\item CLIC seems like a compromise between interesting physics and feasibility
\item There would be more to write about CLIC
\item We mostly agree on waiting
\item However there is already a large delay as the collider is being built
\item What if we wait and no new physics appears?
\end{itemize}

\begin{tabularx}{\textwidth}{X p{4.5cm} p{1.2cm}}
  \textbf{Action Items} & \textbf{Persons Responsible} & \textbf{Deadline} \\

Think about whether CLIC would be a good idea, and reasons why/why not
& Everyone &
18/02/14 \\

\end{tabularx}

\subsubsection{Conclusions}

\begin{tabularx}{\textwidth}{X p{4.5cm} p{1.2cm}}
  \textbf{Action Items} & \textbf{Persons Responsible} & \textbf{Deadline} \\

Upload notes on Group 10 to Basecamp
& Josh \& Alex &
12/02/14 \\

Group meeting to discuss review at 11am after Nuclear \& Particle Physics
&Everyone & 
14/02/14\\

Comparison table
& Esther &
18/02/14 \\

\end{tabularx}

\subsection{Meeting Five}

\begin{tabular}{l l}
  Date & February 18th 2014, 12:00 \\
  Location & Physics E1 \\
  Meeting Called by & Harapan Ong \\
  Type of Meeting & Discussion \\
  Secretary & Esther \\
  Attendees & Dr. Deppisch, Harapan, Josh, Alex, Mahera, \\ & Esther, Tom, Mary, Tim, Izzy \\
  Absent & N/A \\  
\end{tabular}


\subsubsection{Meeting with Prof. Butterworth - Mary \& Harapan}

\begin{itemize}
\item We should aim for a larger Physics reach
\item  ILC is well suited to probe Higgs but not suited for discovery beyond the Standard Model
\item  Dr Deppisch  \textemdash \, As Butterworth is part of the scientific community working to investigate the Standard Model we should take his advice with a pinch of salt
\item  There is a strong case for ILC: Higgs coupling
\item  There is a case to made for building soon  \textemdash \, a generation of scientists could go by without a new collider
\item  There are huge power issues with TLEP and CLIC
\item  Particle physics does not just consist of colliders, the field would not collapse if we were to wait
\item  Smaller projects without international collaboration are still valid
\item  Does not think CERN could cope with both LHC and CLIC, and this would be too focused on Europe
\item  Photon collider addition to ILC  \textemdash \, not advised. Looks good in theory
\item  Izzy - Would reach more physics as initial coupling is to pair of photons. Could reveal heavier particles which Higgs would not decay directly into. Would cost approximately \$1billion
\end{itemize}



\subsubsection{Comparison of Colliders - Esther}

\begin{itemize}
\item Need extra information on the physics and particles which could possibly be found
\item  Most spin-offs were similar irrespective of collider
\item  We should analyse funding models, governance, politics and impact on future projects
\item  Could upload comparison as a Google Doc for everyone to edit
\end{itemize}

\begin{tabularx}{\textwidth}{X p{4.5cm} p{1.2cm}}
  \textbf{Action Items} & \textbf{Persons Responsible} & \textbf{Deadline} \\

Upload comparison to Google Documents
& Esther &
25/02/14 \\

\end{tabularx}


\subsubsection{Choice of Collider - Everyone}

\begin{itemize}
\item Penalty for waiting  \textemdash \, the community of physicists would only be working on LHC, stops pushing the community forward
\item Tom  \textemdash \, choosing ILC may mean there is less for international community to pay as Japan would pay \$1.8billion
\item Argument for collider is always to push the energy boundaries/limit as there is no real upper limit for supersymmetric particles, extra dimensions
\item Mary  \textemdash \, We cannot disprove supersymmetry as we do not know exactly at which energy range to find it
\item We could investigate the intensity frontier as this is suited to precision
\item Dr Deppisch  \textemdash \, Personally inclined to push energy frontier
\item How will CLIC push energy frontier? LHC brings a spectrum of energies, we do not know the exact energy of each collision whereas with CLIC and ILC we can investigate specific energies
\item CLIC will expand the energy frontier, maybe not much in comparison to LHC. However is this a sufficient reason to choose CLIC?
\item We could write a shortlist or ranking
\item Why not choose 2 colliders, compare them in depth for the majority of the report, then choose 1 at the end
\item Are ILC and CLIC the two strongest colliders?
\end{itemize}

\begin{tabularx}{\textwidth}{X p{4.5cm} p{1.2cm}}
  \textbf{Action Items} & \textbf{Persons Responsible} & \textbf{Deadline} \\

STRUCTURE PROJECT \& DELEGATE ROLES
& Everyone &
 \\

\end{tabularx}

\subsubsection{Conclusion}

\begin{itemize}
\item Introduction: Why we are focusing on lepton colliders, why we have chosen ILC and CLIC - Esther
\item  Physics Goals/Reach  \textemdash \, Tom \& Alex
\item  Spin Offs / Timescale  \textemdash \, Mahera
\item  Feasibility  \textemdash \, Harapan \& Josh
\item  Cost / Governance / Location  \textemdash \, Tim \& Izzy
\item  Critical Assessment: Is what ILC and CLIC claim realistic? Suggestions on how to improve existing designs by looking at past designs and where they have failed - Mary
\end{itemize}

\begin{tabularx}{\textwidth}{X p{4.5cm} p{1.2cm}}
  \textbf{Action Items} & \textbf{Persons Responsible} & \textbf{Deadline} \\

Rough draft of your section
& Everyone &
25/02/14 \\

Completed section
& Everyone &
04/03/14 \\

\end{tabularx}

\subsection{Meeting Six}

\begin{tabular}{l l}
  Date & February 25th 2014, 12:00 \\
  Location & G12, South Wing \\
  Meeting Called by & Harapan Ong \\
  Type of Meeting & Discussion \\
  Secretary & Mahera \& Esther \\
  Attendees & Harapan, Mahera, Esther, Josh, Alex, Mary, Tim, Izzy \\
  Absent & Tom (Interview) \\  
\end{tabular}

\subsubsection{Presenting Work}

\begin{itemize}
\item \textbf{Tim  \textemdash \, Cost and funding}
\item No access to funding models for CLIC.
\item There is already a cost report for ILC
\item Need to critically analyse cost reports rather than just quoting
\item The fact that there is no report for CLIC is a drawback in itself
\item We should see whether the current ILC funding model is the best: look how progressed the model is, i.e. does it have government approval?
\item \textbf{Josh - Feasibility}
\item ILC feasibility does not have as much to say in comparison to CLIC
\item The photon problem with ILC  \textemdash \, document: should we reference it in report?
\item Group decided to just talk about electron-positron but maybe if we decide on ILC we can talk about it
\item The photon problem should not come under feasibility
\item \textbf{Mahera  \textemdash \, Spin Offs}
\item CLIC and ILC have the same spin offs, so it’s only 5 pages.
\item Page count doesn’t matter  \textemdash \, group consensus
\item New technologies from CLIC? Will CLIC push technology further? Especially with its higher acceleration gradient
\item Just needs to be a critical assessment
\item \textbf{Alex  \textemdash \, Goals and Reach}
\item We should do overlapping goals for both ILC and CLIC and compare them.
\item Is 1TeV all we need?
\item Possibly include timescale
\item \textbf{Conclusions}
\item There’s a clear science goal for ILC, not so much for CLIC
\item It is hard to galvanise a project when the aims may not be achievable and are unclear
\item Supersymmetric plausibility is not great. It is becoming increasingly unlikely that it is true.
\item Unknowns may only be discovered at LHC and probably won’t be found
\item Unknowns may only be investigated at CLIC, however we need a solid science aim, not just an unknown
\item We should go for CLIC if we can find a compelling argument. Tom \& Alex’s goals section is very important for this.
\end{itemize}

\begin{tabularx}{\textwidth}{X p{4.5cm} p{1.2cm}}
  \textbf{Action Items} & \textbf{Persons Responsible} & \textbf{Deadline} \\

Finish reports comparing ILC and CLIC
& Everyone &
04/03/14 \\

\end{tabularx}

\subsubsection{Presentation}

\begin{itemize}
\item Pitch for funding for the collider
\item Presentation should not be a comparison between ILC and CLIC
\item Maybe  \textemdash \, limit the amount of time spent on the other collider
\item Prezi could be used
\item Talk about LHC as a success case
\item Talk about others as a contrast in different sections, e.g. timescale to show that we have researched them

\end{itemize}

\subsubsection{Conclusions}

\begin{tabularx}{\textwidth}{X p{4.5cm} p{1.2cm}}
  \textbf{Action Items} & \textbf{Persons Responsible} & \textbf{Deadline} \\

Finish off comparison reports
& Everyone &
04/03/14 \\
 
 If we write anything which is a physical explanation, send to Alex
& Everyone &
N/A \\
 
 Start Latex compilation next week
& Alex \& Mary &
Start 04/03/14 \\

\end{tabularx}


\subsection{Meeting Seven}

\begin{tabular}{l l}
  Date & March 4th 2014, 12:00 \\
  Location & SM1, Cruciform \\
  Meeting Called by & Dr. Frank Deppisch \\
  Type of Meeting & Discussion \\
  Secretary & Mahera \& Esther \\
  Attendees & Dr. Frank Deppisch (Skype), Harapan, Mahera, \\
  &  Esther, Josh, Alex, Tom, Mary, Tim, Izzy \\
  Absent & N/A \\  
\end{tabular}

\begin{itemize}
\item If we are all 50/50 the decision could be based on excitement i.e. for the public or the scientific community
Funding
\item  CLIC has no funding model available. We have to base our estimates on something, therefore we need to take this into account; we can only quote what we have been given
\item  Minutes state a cost of £6billion but we need a critical assessment of this cost
\item \textbf{Location}
\item  CLIC would be built at CERN which is “safe” whereas Japan needs to create infrastructure. However lots of research has been carried out for ILC
\item  ILC has a more detailed plan. Can CERN handle CLIC?
\item  We cannot have both CLIC and VLHC.
\item  CERN has a certain (200MW?) power budget, which means they would have to expand their power budget to accommodate CLIC
\item  ILC is dependent on Japan’s economy, which makes it less stable. A host country has to be 100\% behind the project.
\item  Location seems to be a stalemate
\item \textbf{Timeline}
\item  ILC construction should be finished by 2026 (10 year construction time)
\item  Commissioning of CLIC goes beyond 2030
\item  Therefore ILC wins this; it is fast so keeps the next generation of physicists excited. The longer we wait, the shorter the running time to research
\item \textbf{Feasibility}
\item  A lot less R\&D to be done for ILC, e.g. acceleration gradient
\item  ILC has demonstrated 35MV/m in the lab but not on an industrial scale
\item  CLIC has more research to be done, whereas ILC’s needs are more about mass production and  industrial scale, e.g. Marx Modulators
\item \textbf{Reach}
\item  CLIC wins as it pushes the energy frontier as well as being precise
\item  Higher energies are needed in order to probe supersymmetric particles
\item  However supersymmetry, leptoquarks etc all seem unlikelu
\item  There does not seem to be a strong science case. The higher energy of CLIC does not seem to be justifiable which means we cannot make a case for it.
\item  There have not been any suspicious events at higher energies at LHC to justify investigating further
\end{itemize}

\textbf{We choose ILC}

\begin{tabularx}{\textwidth}{X p{4.5cm} p{1.2cm}}
  \textbf{Action Items} & \textbf{Persons Responsible} & \textbf{Deadline} \\

Executive summary: Like an abstract, what we’ve found and conclusions, a summary of our whole report. ~1 page at the start
& Izzy &
11/03/14 \\

Conclusion: Summary of just ILC and CLIC. Including table, ~3 pages
& Esther &
11/03/14 \\

\end{tabularx}

\subsubsection{Conclusions}

\begin{tabularx}{\textwidth}{X p{4.5cm} p{1.2cm}}
  \textbf{Action Items} & \textbf{Persons Responsible} & \textbf{Deadline} \\

Presentation: Prezi; based on report. Meeting in Physics Common Room
& Everyone &
07/03/14 \\

Collate work on Latex
& Alex \& Mary &
07/03/14 \\
 
 Read collated reports
& Everyone &
11/03/14 \\

Executive summary
& Izzy &
11/03/14 \\

Conclusions
& Esther  &
11/03/14 \\

\end{tabularx}

\subsection{Meetings Eight \& Nine}

12:00, 11th and 18th of March 2014.

Presentation Practice and Discussion.

\subsection{Meeting Ten}

13:00, 25th March 2014. B03, 16 Gordon Square.

Presentation Rehearsal with Dr. Frank Deppisch
