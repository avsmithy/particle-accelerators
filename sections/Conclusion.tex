In conclusion, we have chosen ILC as the next future collider. We feel that it is the best option as it combines both feasibility in terms of location, cost and technology, and physics reach.

\subsection{Location}

The proposed location of CLIC was CERN, whereas ILC's main contender for location is in Japan. A concern for our group was that if a new collider was to be built at CERN, its energy budget would have to be expanded to accommodate this. CERN already has an infrastructure, in contrast to ILC where infrastructure would need to be created. However plenty of research has been carried out to ensure that Kitakami in northern Japan, the chosen location for ILC, is safe, including topography, geology, drainage and earthquake studies, as well as examining transport links and nearby major cities.

\subsection{Finance}

In terms of finance, ILC has a highly developed financial plan, while CLIC's funding model is not finalised and is currently inaccessible. The cost of ILC has been estimated at \$7.8 bn. As the research has been carried out into the exact costs of ILC and a sub-system value breakdown has been given, this estimate is more reliable than the estimates of \$6 billion found for CLIC.

\subsection{Spin Offs}

The spin offs that both particle accelerators would lead to are very similar. From improved medical linear acceleration, smaller beam radius for radiotherapy uses and even nuclear waste disposal, both accelerators would achieve these aims. Due to CLIC's higher acceleration gradient, power consumption and the size of the equipment for hospital linear accelerators could be greatly reduced. However, there is no evidence to suggest that this would not be the case for ILC. In addition, because ILC could be built within a shorter timescale, it could have the added spin off of attracting people and skills to the area where the accelerator is built, and ensuring the international community of scientists is pushed forward.

\subsection{Energy Reach}

CLIC undeniably has a higher energy reach than ILC: 3TeV in comparison to 1TeV. However, the question we found ourselves asking was, ``is this higher energy necessary?''  One argument was that pushing the energy frontier is always desirable, yet the evidence for particles at higher energies was lacking at best. For instance, both CLIC and ILC had goals of discovering supersymmetric particles. However, the existence of supersymmetry is doubtful and is becoming increasingly unlikely. In addition, LHC has not found any evidence of supersymmetry at higher energies which we feel would be needed to justify the higher energy range of CLIC. ILC seems to have a clear science goal which is not the case for CLIC. It would be hard to galvanise a project such as CLIC where its aims may not be achievable. We found that the risk of committing to a large scale project such as CLIC with no little evidence of the particles the higher energy could be used to investigate, was not attractive.

\subsection{Feasibility}

Both ILC and CLIC could be used to probe Higgs in more detail, therefore due to the higher feasibility of the former, it seems as though ILC is the clear choice.

The LCC is currently focusing on possible construction for ILC, in comparison to the focus on research for CLIC. Therefore in the international physics community ILC is already seen as closer to construction, and more feasible than CLIC. CLIC still needs the technology for its two beam linear accelerators and low beam emittance to progress before construction can occur, although these are in their final stages of development. Instead of waiting for the technology and high acceleration gradient to be developed for CLIC, we have decided to ride the momentum of the LHC's Higgs discovery and keep the next generation of physicists engaged by building ILC. With the cleaner collisions of a lepton collider in comparison to LHC, the Higgs can be probed sooner rather than later by deciding on ILC, which will galvanise current interest in the Higgs boson in the scientific community and in the press.